%================================================ CHAPITRE 1 ===============================================

\chapter{ÉTUDE DE L'EXISTANT ET GÉNÉRALITÉ DES CONCEPTS}
\section{Introduction partielle}
\paragraph{}
Ce chapitre inaugural plonge au cœur de notre étude en dressant un état des lieux 
complet des systèmes existants et en posant les bases 
conceptuelles essentielles à la compréhension de notre travail

Nous débuterons par une présentation détaillée de l’entreprise, etc.

\section{Étude de l’existant \& Généralité des concepts}
\subsection{Présentation de l’entreprise XXXXX}
        \subsubsection{Cadre historique}
        \paragraph{}
        Une présentation d'un point de vue historique de l'entreprise(ou des entreprises).
        \subsubsection{Cadre géographique}
        \paragraph{}
        Une présentation d'un point de vue géographique de l'entreprise(ou des entreprises).
        \subsubsection{Cadre organisationnel}
        \paragraph{}
        Une présentation d'un point de vue organisationnel de l'entreprise(ou des entreprises).

\subsection{Étude de l'existant}
    L'étude de l'existant dans un travail scientifique consiste à faire un état des lieux détaillé du domaine d'étude 
    avant d'entreprendre une nouvelle recherche, afin de comprendre le contexte, identifier les forces et faiblesses du 
    système actuel, recenser les objectifs existants et les contraintes, et ainsi délimiter clairement le champ de 
    l'investigation pour définir les besoins et éviter les doublons, utilisant des méthodes comme l'interview ou 
    l'enquête.

\subsection{Critique de l'existant}
\paragraph{}
    \subsubsection{Points forts}
    \paragraph{}
    Les points forts du système existants.
    

    \subsubsection{Points faibles}
    \paragraph{}
    Les points faibles du système existants.
\subsection{Besoins fonctionnels \& non fonctionnels}
\subsubsection{Besoins fonctionnels}
\paragraph{}
Les besoins dont vous avez trouvez fonctionnels au sein de l'entreprise
\subsubsection{Besoins non fonctionnels}
\paragraph{}
Les besoins dont vous avez trouvez non fonctionnels au sein de l'entreprise.

\section{Conclusion partielle}
Une conclusion partielle de ce chapitre, suivie d'une transition vers le chapitre suivant.