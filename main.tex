%====================================================================
%=  Author : André MUTOKE                                           =
%=  file : main.tex                                                 =
%=  Description : Ce fichier contient la strcutre principale de     =
%=                votre travail de fin de cycle suivant le          =
%=                modèle de l'Universite Don Bosco de Lubumbashi    =
%=                (UDBL)                                            =
%====================================================================
\documentclass[12pt, a4paper]{report}
\usepackage{fancyhdr}                      % Pour definir les entetes et les pieds de page
\usepackage[T1]{fontenc}                   % Encodage T1 (adapté au français)
\usepackage{lmodern}                       % Caractères plus lisibles
\usepackage[french,english]{babel}         % Réglages linguistiques (avec french)
\usepackage{zref-perpage}                  % Pour reinitialiser le compteur des pages
\usepackage{graphicx}                      % Pour le graphic
\usepackage{url}                           % Pour gerer l'url
\usepackage[utf8]{inputenc}                          
\usepackage{ragged2e}
\usepackage{tabularx}
\usepackage[toc,page]{appendix}
%\usepackage[nottoc]{tocbibind}
\usepackage{natbib}
\usepackage{microtype} % makes pdf look better
\usepackage[caption=false]{subfig}
\usepackage{acronym}

\usepackage[left=35mm, top=35mm, right=25mm, bottom=25mm]{geometry} % Bordure des pages selon le modele de UDBL
\usepackage{setspace} \renewcommand{\baselinestretch}{1.25}
\usepackage{titlesec}

\usepackage{listings}
\usepackage{listingsutf8}
\usepackage{color}
\usepackage{pdfpages} % Pour permettre d'inserer des pages des pdf en annexe


\setlength{\parskip}{1.5ex plus 0.5ex minus 0.5ex} % Espacement entre les paragraphes
% \setlength{\parindent}{2.5em}

%%%%%%%%% GESTION DES BLOC POUR CONTENIR DES CODES EN C++
\definecolor{dkgreen}{rgb}{0,0.6,0}
\definecolor{gray}{rgb}{0.5,0.5,0.5}
\definecolor{mauve}{rgb}{0.58,0,0.82}

\lstset{
  language=C++,
  basicstyle=\ttfamily\footnotesize,
  numbers=left,
  numberstyle=\tiny\color{gray},
  stepnumber=1,
  numbersep=5pt,
  backgroundcolor=\color{white},
  showspaces=false,
  showstringspaces=false,
  showtabs=false,
  frame=single,
  tabsize=2,
  captionpos=b,
  breaklines=true,
  breakatwhitespace=false,
  title=\lstname,
  keywordstyle=\color{dkgreen},
  commentstyle=\color{gray},
  stringstyle=\color{blue},
  escapeinside={\%*}{*)},
  morekeywords={typename, class, template, constexpr, auto}
}

\usepackage{amsmath}

% ============================================ DESIGN DES SECTIONS / SOUS SECTION et SOUS SOUS SECTION ===========================================================

\titlespacing{\section}{0pt}{3pt}{3pt}

\titleformat % design des titres des sections
{\section}
[block]
{\normalsize\bfseries}
{\thesection~.}
{\baselineskip}
{}
[]

% ==============================================================================================================================

\titleformat % design des titres des sous-sections
{\subsection}
[block]
{\normalfont\bfseries}
{\thesubsection~.}
{\baselineskip}
{}
[]

% ======================================================================================================================
\titleformat % design des titres des sous-sous-sections
{\subsubsection}
[block]
{\itshape\normalsize\bfseries}
{\normalfont\bfseries \thesubsubsection~.}
{\baselineskip}
{}
[]
%=============================================== FIN DU PARAMETRAGE DES SECTIONS ===========================================

%=============================================== DESIGN DES CHAPITRES ======================================================
% Des redéfinitions des chapitres
\makeatletter
\def\@makechapterhead#1{%
    
    \vspace*{0.5\p@}%
    {
        \parindent \z@ \raggedright \normalfont
        \interlinepenalty\@M
        \vskip 2\p@
        \rule{1\linewidth}{1.5pt} \vskip 2\p@
        \ifnum \c@secnumdepth >\m@ne % Les chapitre numerotés
            \bfseries CHAP \thechapter\quad
        \fi
        \centering \bfseries #1\par\nobreak 
        \vskip 1\p@
        \rule{1\linewidth}{1pt}
        \vskip 20\p@
            
    }
}

\def\@makeschapterhead#1{%
    \begin{center}
        \vspace*{0.5\p@}%
        {\parindent \z@ \raggedright
            \normalfont
            \interlinepenalty\@M % Les chapitres non numérotés
            \vskip 2\p@
            \rule{1\linewidth}{1.5pt} \vskip 2\p@
            \centering \bfseries  #1\par\nobreak
            \vskip 1\p@
            \rule{1\linewidth}{1pt}
            \vskip 20\p@
        }
    \end{center}
}
\makeatother

% =========================================== FIN CONFIGURATION DES CHAPITRES ===============================================
%============================================ REDEFINITONS DE CERTAINES COMMANDES SUPLEMENTAIRES ============================
% Redefinition de certaines commandes
% \addto\captionsenglish{
% \renewcommand{\chaptername}{Chapitre}
% }
\addto\captionsenglish{
\renewcommand{\listfigurename}{LISTE DES FIGURES}
}
\addto\captionsenglish{
\renewcommand{\listtablename}{LISTE DES TABLEAUX}
}
\addto\captionsenglish{
\renewcommand{\contentsname}{TABLE DES MATIÈRES}
}
\addto\captionsenglish{
\renewcommand{\contentsname}{TABLE DES MATIÈRES}
}

\addto\captionsenglish{
\renewcommand{\bibname}{BIBLIOGRAPHIE}
}


% Page de garde
% =============================================== PAGE I ===================================================================
% POUR UTILISER LA PAGE DE GARDE UDBL, UTILISER DANS VOTRE FICHIER DE TRAVAIL
% 1. TELECHARGE SUR LE GITHUB, LE PROJET DANS VOTRE DOSSIER DE TRAVAIL
% 2. INCLURE LE FICHIER A VOTRE PROJET COMME INDIQUE CI-DESSOUS
%         \documentclass{artcle}
%         \usepackage[utf8]{inputenc}
%        
%         % =============================================== PAGE I ===================================================================
% POUR UTILISER LA PAGE DE GARDE UDBL, UTILISER DANS VOTRE FICHIER DE TRAVAIL
% 1. TELECHARGE SUR LE GITHUB, LE PROJET DANS VOTRE DOSSIER DE TRAVAIL
% 2. INCLURE LE FICHIER A VOTRE PROJET COMME INDIQUE CI-DESSOUS
%         \documentclass{artcle}
%         \usepackage[utf8]{inputenc}
%        
%         % =============================================== PAGE I ===================================================================
% POUR UTILISER LA PAGE DE GARDE UDBL, UTILISER DANS VOTRE FICHIER DE TRAVAIL
% 1. TELECHARGE SUR LE GITHUB, LE PROJET DANS VOTRE DOSSIER DE TRAVAIL
% 2. INCLURE LE FICHIER A VOTRE PROJET COMME INDIQUE CI-DESSOUS
%         \documentclass{artcle}
%         \usepackage[utf8]{inputenc}
%        
%         \input{latex_page_garde/page_garde}
%         \begin{document}
%            Mon document .........
%         \end{document}
%
% 3. UTILISER LA COMMANDE SUIVANTE DANS VOTRE TRAVAIL EN SUIVANT LA SYNTAXE SUIVANTE :
%
%      \pagedegardeudbl{TITRE DU DOCUMENT}
%                      {NOM DE L'AUTEUR}
%                      {NOM DU DIRECTEUR}
%                      {NOM DU CO-DIRECTEUR}
%                      {NOM DE LA FILIERE ET PROMOTION}
%
%
% 4. VOICI L'EXEMPLE FINAL ET COMPLET
%
% 
%
%         \documentclass{book}
%         \usepackage[utf8]{inputenc}
%         \usepackage[left=2cm, right=2cm, top=2cm, bottom=2cm]{geometry}
%         \usepackage{ragged2e}
%         \usepackage{graphicx}
%         \usepackage{url}
%
%         \input{latex_page_garde/page_garde}
%         \begin{document}
%            \pagedegardeudbl{Migration de la telephonie classique vers la ToIP}
%                            {André MUTOKE}
%                            {Emile MUTOKE}
%                            {Baudouin BANZA}
%                            {L4 TLC et Reseaux}
%         \end{document}
\newcommand{\pagedegardeudbl}[5]{
    \pagenumbering{gobble}
    \begin{center}
        UNIVERSIT\'E DON BOSCO DE LUBUMBASHI \\
        Faculté des Sciences Informatiques \\
        Département Réseaux\\
        Lubumbashi \\
        www.udbl.ac.cd \\
    \end{center}
    \begin{center}
        \rule{\linewidth}{2pt}
    \end{center}

    \vspace{10pt}
    \begin{center}
        \includegraphics[scale=.6]{latex_page_garde/udbl.png}
    \end{center}

    \begin{center}
        \rule{\linewidth}{1pt}
    \end{center}
    \begin{center}
        \textbf{\large #1 \\
        (\emph{cas de MMG et de l'ARPTC})
        }
    \end{center}
    \begin{center}
        \rule{\linewidth}{1pt}
    \end{center}

    \vspace{5pt}

    \begin{FlushRight}
        \emph{Travail présenté et défendu en vue de l’obtention  \\  du grade d’ingénieur technicien en Informatique} \\
        \vspace{5pt}
        Présenté par :\textbf{#2} \\
        Filière : \textbf{#5} \\
    \end{FlushRight}
    \vspace{5pt}
    \begin{center}
        \textbf{\large Août 2025}
    \end{center}
    \newpage

    % =============================================== PAGE II ===================================================================

    \begin{center}
        UNIVERSIT\'E DON BOSCO DE LUBUMBASHI \\
        Faculté des Sciences Informatiques \\
        Département Réseaux\\
        Lubumbashi \\
        www.udbl.ac.cd \\
    \end{center}

    \begin{center}
        \rule{\linewidth}{2pt}
    \end{center}
    % ++++++++++++++++++++++++++++++ LE LOGO + LE TITRE
    \vspace{10pt}
    \begin{center}
        \includegraphics[scale=.6]{latex_page_garde/udbl.png}
    \end{center}

    \begin{center}
        \rule{\linewidth}{1pt}
    \end{center}
    \begin{center}
        \textbf{\large #1 \\
        (\emph{cas de MMG et de l'ARPTC})
        }
    \end{center}
    \begin{center}
        \rule{\linewidth}{1pt}
    \end{center}

    \vspace{5pt}

    \begin{FlushRight}
        \emph{Travail présenté et défendu en vue de l’obtention \\ du grade d’ingénieur technicien en Informatique} \\
        \vspace{5pt}
        Présenté par : \textbf{#2} \\
        Filière : \textbf{#5} \\
        \vspace{5pt}
        Directeur : \textbf{#3} \\
        Co-directeur : \textbf{#4} \\
    \end{FlushRight}
    \vspace{5pt}
    \begin{center}
        \textbf{\large Août 2025}
    \end{center}
}
\newpage

%         \begin{document}
%            Mon document .........
%         \end{document}
%
% 3. UTILISER LA COMMANDE SUIVANTE DANS VOTRE TRAVAIL EN SUIVANT LA SYNTAXE SUIVANTE :
%
%      \pagedegardeudbl{TITRE DU DOCUMENT}
%                      {NOM DE L'AUTEUR}
%                      {NOM DU DIRECTEUR}
%                      {NOM DU CO-DIRECTEUR}
%                      {NOM DE LA FILIERE ET PROMOTION}
%
%
% 4. VOICI L'EXEMPLE FINAL ET COMPLET
%
% 
%
%         \documentclass{book}
%         \usepackage[utf8]{inputenc}
%         \usepackage[left=2cm, right=2cm, top=2cm, bottom=2cm]{geometry}
%         \usepackage{ragged2e}
%         \usepackage{graphicx}
%         \usepackage{url}
%
%         % =============================================== PAGE I ===================================================================
% POUR UTILISER LA PAGE DE GARDE UDBL, UTILISER DANS VOTRE FICHIER DE TRAVAIL
% 1. TELECHARGE SUR LE GITHUB, LE PROJET DANS VOTRE DOSSIER DE TRAVAIL
% 2. INCLURE LE FICHIER A VOTRE PROJET COMME INDIQUE CI-DESSOUS
%         \documentclass{artcle}
%         \usepackage[utf8]{inputenc}
%        
%         \input{latex_page_garde/page_garde}
%         \begin{document}
%            Mon document .........
%         \end{document}
%
% 3. UTILISER LA COMMANDE SUIVANTE DANS VOTRE TRAVAIL EN SUIVANT LA SYNTAXE SUIVANTE :
%
%      \pagedegardeudbl{TITRE DU DOCUMENT}
%                      {NOM DE L'AUTEUR}
%                      {NOM DU DIRECTEUR}
%                      {NOM DU CO-DIRECTEUR}
%                      {NOM DE LA FILIERE ET PROMOTION}
%
%
% 4. VOICI L'EXEMPLE FINAL ET COMPLET
%
% 
%
%         \documentclass{book}
%         \usepackage[utf8]{inputenc}
%         \usepackage[left=2cm, right=2cm, top=2cm, bottom=2cm]{geometry}
%         \usepackage{ragged2e}
%         \usepackage{graphicx}
%         \usepackage{url}
%
%         \input{latex_page_garde/page_garde}
%         \begin{document}
%            \pagedegardeudbl{Migration de la telephonie classique vers la ToIP}
%                            {André MUTOKE}
%                            {Emile MUTOKE}
%                            {Baudouin BANZA}
%                            {L4 TLC et Reseaux}
%         \end{document}
\newcommand{\pagedegardeudbl}[5]{
    \pagenumbering{gobble}
    \begin{center}
        UNIVERSIT\'E DON BOSCO DE LUBUMBASHI \\
        Faculté des Sciences Informatiques \\
        Département Réseaux\\
        Lubumbashi \\
        www.udbl.ac.cd \\
    \end{center}
    \begin{center}
        \rule{\linewidth}{2pt}
    \end{center}

    \vspace{10pt}
    \begin{center}
        \includegraphics[scale=.6]{latex_page_garde/udbl.png}
    \end{center}

    \begin{center}
        \rule{\linewidth}{1pt}
    \end{center}
    \begin{center}
        \textbf{\large #1 \\
        (\emph{cas de MMG et de l'ARPTC})
        }
    \end{center}
    \begin{center}
        \rule{\linewidth}{1pt}
    \end{center}

    \vspace{5pt}

    \begin{FlushRight}
        \emph{Travail présenté et défendu en vue de l’obtention  \\  du grade d’ingénieur technicien en Informatique} \\
        \vspace{5pt}
        Présenté par :\textbf{#2} \\
        Filière : \textbf{#5} \\
    \end{FlushRight}
    \vspace{5pt}
    \begin{center}
        \textbf{\large Août 2025}
    \end{center}
    \newpage

    % =============================================== PAGE II ===================================================================

    \begin{center}
        UNIVERSIT\'E DON BOSCO DE LUBUMBASHI \\
        Faculté des Sciences Informatiques \\
        Département Réseaux\\
        Lubumbashi \\
        www.udbl.ac.cd \\
    \end{center}

    \begin{center}
        \rule{\linewidth}{2pt}
    \end{center}
    % ++++++++++++++++++++++++++++++ LE LOGO + LE TITRE
    \vspace{10pt}
    \begin{center}
        \includegraphics[scale=.6]{latex_page_garde/udbl.png}
    \end{center}

    \begin{center}
        \rule{\linewidth}{1pt}
    \end{center}
    \begin{center}
        \textbf{\large #1 \\
        (\emph{cas de MMG et de l'ARPTC})
        }
    \end{center}
    \begin{center}
        \rule{\linewidth}{1pt}
    \end{center}

    \vspace{5pt}

    \begin{FlushRight}
        \emph{Travail présenté et défendu en vue de l’obtention \\ du grade d’ingénieur technicien en Informatique} \\
        \vspace{5pt}
        Présenté par : \textbf{#2} \\
        Filière : \textbf{#5} \\
        \vspace{5pt}
        Directeur : \textbf{#3} \\
        Co-directeur : \textbf{#4} \\
    \end{FlushRight}
    \vspace{5pt}
    \begin{center}
        \textbf{\large Août 2025}
    \end{center}
}
\newpage

%         \begin{document}
%            \pagedegardeudbl{Migration de la telephonie classique vers la ToIP}
%                            {André MUTOKE}
%                            {Emile MUTOKE}
%                            {Baudouin BANZA}
%                            {L4 TLC et Reseaux}
%         \end{document}
\newcommand{\pagedegardeudbl}[5]{
    \pagenumbering{gobble}
    \begin{center}
        UNIVERSIT\'E DON BOSCO DE LUBUMBASHI \\
        Faculté des Sciences Informatiques \\
        Département Réseaux\\
        Lubumbashi \\
        www.udbl.ac.cd \\
    \end{center}
    \begin{center}
        \rule{\linewidth}{2pt}
    \end{center}

    \vspace{10pt}
    \begin{center}
        \includegraphics[scale=.6]{latex_page_garde/udbl.png}
    \end{center}

    \begin{center}
        \rule{\linewidth}{1pt}
    \end{center}
    \begin{center}
        \textbf{\large #1 \\
        (\emph{cas de MMG et de l'ARPTC})
        }
    \end{center}
    \begin{center}
        \rule{\linewidth}{1pt}
    \end{center}

    \vspace{5pt}

    \begin{FlushRight}
        \emph{Travail présenté et défendu en vue de l’obtention  \\  du grade d’ingénieur technicien en Informatique} \\
        \vspace{5pt}
        Présenté par :\textbf{#2} \\
        Filière : \textbf{#5} \\
    \end{FlushRight}
    \vspace{5pt}
    \begin{center}
        \textbf{\large Août 2025}
    \end{center}
    \newpage

    % =============================================== PAGE II ===================================================================

    \begin{center}
        UNIVERSIT\'E DON BOSCO DE LUBUMBASHI \\
        Faculté des Sciences Informatiques \\
        Département Réseaux\\
        Lubumbashi \\
        www.udbl.ac.cd \\
    \end{center}

    \begin{center}
        \rule{\linewidth}{2pt}
    \end{center}
    % ++++++++++++++++++++++++++++++ LE LOGO + LE TITRE
    \vspace{10pt}
    \begin{center}
        \includegraphics[scale=.6]{latex_page_garde/udbl.png}
    \end{center}

    \begin{center}
        \rule{\linewidth}{1pt}
    \end{center}
    \begin{center}
        \textbf{\large #1 \\
        (\emph{cas de MMG et de l'ARPTC})
        }
    \end{center}
    \begin{center}
        \rule{\linewidth}{1pt}
    \end{center}

    \vspace{5pt}

    \begin{FlushRight}
        \emph{Travail présenté et défendu en vue de l’obtention \\ du grade d’ingénieur technicien en Informatique} \\
        \vspace{5pt}
        Présenté par : \textbf{#2} \\
        Filière : \textbf{#5} \\
        \vspace{5pt}
        Directeur : \textbf{#3} \\
        Co-directeur : \textbf{#4} \\
    \end{FlushRight}
    \vspace{5pt}
    \begin{center}
        \textbf{\large Août 2025}
    \end{center}
}
\newpage

%         \begin{document}
%            Mon document .........
%         \end{document}
%
% 3. UTILISER LA COMMANDE SUIVANTE DANS VOTRE TRAVAIL EN SUIVANT LA SYNTAXE SUIVANTE :
%
%      \pagedegardeudbl{TITRE DU DOCUMENT}
%                      {NOM DE L'AUTEUR}
%                      {NOM DU DIRECTEUR}
%                      {NOM DU CO-DIRECTEUR}
%                      {NOM DE LA FILIERE ET PROMOTION}
%
%
% 4. VOICI L'EXEMPLE FINAL ET COMPLET
%
% 
%
%         \documentclass{book}
%         \usepackage[utf8]{inputenc}
%         \usepackage[left=2cm, right=2cm, top=2cm, bottom=2cm]{geometry}
%         \usepackage{ragged2e}
%         \usepackage{graphicx}
%         \usepackage{url}
%
%         % =============================================== PAGE I ===================================================================
% POUR UTILISER LA PAGE DE GARDE UDBL, UTILISER DANS VOTRE FICHIER DE TRAVAIL
% 1. TELECHARGE SUR LE GITHUB, LE PROJET DANS VOTRE DOSSIER DE TRAVAIL
% 2. INCLURE LE FICHIER A VOTRE PROJET COMME INDIQUE CI-DESSOUS
%         \documentclass{artcle}
%         \usepackage[utf8]{inputenc}
%        
%         % =============================================== PAGE I ===================================================================
% POUR UTILISER LA PAGE DE GARDE UDBL, UTILISER DANS VOTRE FICHIER DE TRAVAIL
% 1. TELECHARGE SUR LE GITHUB, LE PROJET DANS VOTRE DOSSIER DE TRAVAIL
% 2. INCLURE LE FICHIER A VOTRE PROJET COMME INDIQUE CI-DESSOUS
%         \documentclass{artcle}
%         \usepackage[utf8]{inputenc}
%        
%         \input{latex_page_garde/page_garde}
%         \begin{document}
%            Mon document .........
%         \end{document}
%
% 3. UTILISER LA COMMANDE SUIVANTE DANS VOTRE TRAVAIL EN SUIVANT LA SYNTAXE SUIVANTE :
%
%      \pagedegardeudbl{TITRE DU DOCUMENT}
%                      {NOM DE L'AUTEUR}
%                      {NOM DU DIRECTEUR}
%                      {NOM DU CO-DIRECTEUR}
%                      {NOM DE LA FILIERE ET PROMOTION}
%
%
% 4. VOICI L'EXEMPLE FINAL ET COMPLET
%
% 
%
%         \documentclass{book}
%         \usepackage[utf8]{inputenc}
%         \usepackage[left=2cm, right=2cm, top=2cm, bottom=2cm]{geometry}
%         \usepackage{ragged2e}
%         \usepackage{graphicx}
%         \usepackage{url}
%
%         \input{latex_page_garde/page_garde}
%         \begin{document}
%            \pagedegardeudbl{Migration de la telephonie classique vers la ToIP}
%                            {André MUTOKE}
%                            {Emile MUTOKE}
%                            {Baudouin BANZA}
%                            {L4 TLC et Reseaux}
%         \end{document}
\newcommand{\pagedegardeudbl}[5]{
    \pagenumbering{gobble}
    \begin{center}
        UNIVERSIT\'E DON BOSCO DE LUBUMBASHI \\
        Faculté des Sciences Informatiques \\
        Département Réseaux\\
        Lubumbashi \\
        www.udbl.ac.cd \\
    \end{center}
    \begin{center}
        \rule{\linewidth}{2pt}
    \end{center}

    \vspace{10pt}
    \begin{center}
        \includegraphics[scale=.6]{latex_page_garde/udbl.png}
    \end{center}

    \begin{center}
        \rule{\linewidth}{1pt}
    \end{center}
    \begin{center}
        \textbf{\large #1 \\
        (\emph{cas de MMG et de l'ARPTC})
        }
    \end{center}
    \begin{center}
        \rule{\linewidth}{1pt}
    \end{center}

    \vspace{5pt}

    \begin{FlushRight}
        \emph{Travail présenté et défendu en vue de l’obtention  \\  du grade d’ingénieur technicien en Informatique} \\
        \vspace{5pt}
        Présenté par :\textbf{#2} \\
        Filière : \textbf{#5} \\
    \end{FlushRight}
    \vspace{5pt}
    \begin{center}
        \textbf{\large Août 2025}
    \end{center}
    \newpage

    % =============================================== PAGE II ===================================================================

    \begin{center}
        UNIVERSIT\'E DON BOSCO DE LUBUMBASHI \\
        Faculté des Sciences Informatiques \\
        Département Réseaux\\
        Lubumbashi \\
        www.udbl.ac.cd \\
    \end{center}

    \begin{center}
        \rule{\linewidth}{2pt}
    \end{center}
    % ++++++++++++++++++++++++++++++ LE LOGO + LE TITRE
    \vspace{10pt}
    \begin{center}
        \includegraphics[scale=.6]{latex_page_garde/udbl.png}
    \end{center}

    \begin{center}
        \rule{\linewidth}{1pt}
    \end{center}
    \begin{center}
        \textbf{\large #1 \\
        (\emph{cas de MMG et de l'ARPTC})
        }
    \end{center}
    \begin{center}
        \rule{\linewidth}{1pt}
    \end{center}

    \vspace{5pt}

    \begin{FlushRight}
        \emph{Travail présenté et défendu en vue de l’obtention \\ du grade d’ingénieur technicien en Informatique} \\
        \vspace{5pt}
        Présenté par : \textbf{#2} \\
        Filière : \textbf{#5} \\
        \vspace{5pt}
        Directeur : \textbf{#3} \\
        Co-directeur : \textbf{#4} \\
    \end{FlushRight}
    \vspace{5pt}
    \begin{center}
        \textbf{\large Août 2025}
    \end{center}
}
\newpage

%         \begin{document}
%            Mon document .........
%         \end{document}
%
% 3. UTILISER LA COMMANDE SUIVANTE DANS VOTRE TRAVAIL EN SUIVANT LA SYNTAXE SUIVANTE :
%
%      \pagedegardeudbl{TITRE DU DOCUMENT}
%                      {NOM DE L'AUTEUR}
%                      {NOM DU DIRECTEUR}
%                      {NOM DU CO-DIRECTEUR}
%                      {NOM DE LA FILIERE ET PROMOTION}
%
%
% 4. VOICI L'EXEMPLE FINAL ET COMPLET
%
% 
%
%         \documentclass{book}
%         \usepackage[utf8]{inputenc}
%         \usepackage[left=2cm, right=2cm, top=2cm, bottom=2cm]{geometry}
%         \usepackage{ragged2e}
%         \usepackage{graphicx}
%         \usepackage{url}
%
%         % =============================================== PAGE I ===================================================================
% POUR UTILISER LA PAGE DE GARDE UDBL, UTILISER DANS VOTRE FICHIER DE TRAVAIL
% 1. TELECHARGE SUR LE GITHUB, LE PROJET DANS VOTRE DOSSIER DE TRAVAIL
% 2. INCLURE LE FICHIER A VOTRE PROJET COMME INDIQUE CI-DESSOUS
%         \documentclass{artcle}
%         \usepackage[utf8]{inputenc}
%        
%         \input{latex_page_garde/page_garde}
%         \begin{document}
%            Mon document .........
%         \end{document}
%
% 3. UTILISER LA COMMANDE SUIVANTE DANS VOTRE TRAVAIL EN SUIVANT LA SYNTAXE SUIVANTE :
%
%      \pagedegardeudbl{TITRE DU DOCUMENT}
%                      {NOM DE L'AUTEUR}
%                      {NOM DU DIRECTEUR}
%                      {NOM DU CO-DIRECTEUR}
%                      {NOM DE LA FILIERE ET PROMOTION}
%
%
% 4. VOICI L'EXEMPLE FINAL ET COMPLET
%
% 
%
%         \documentclass{book}
%         \usepackage[utf8]{inputenc}
%         \usepackage[left=2cm, right=2cm, top=2cm, bottom=2cm]{geometry}
%         \usepackage{ragged2e}
%         \usepackage{graphicx}
%         \usepackage{url}
%
%         \input{latex_page_garde/page_garde}
%         \begin{document}
%            \pagedegardeudbl{Migration de la telephonie classique vers la ToIP}
%                            {André MUTOKE}
%                            {Emile MUTOKE}
%                            {Baudouin BANZA}
%                            {L4 TLC et Reseaux}
%         \end{document}
\newcommand{\pagedegardeudbl}[5]{
    \pagenumbering{gobble}
    \begin{center}
        UNIVERSIT\'E DON BOSCO DE LUBUMBASHI \\
        Faculté des Sciences Informatiques \\
        Département Réseaux\\
        Lubumbashi \\
        www.udbl.ac.cd \\
    \end{center}
    \begin{center}
        \rule{\linewidth}{2pt}
    \end{center}

    \vspace{10pt}
    \begin{center}
        \includegraphics[scale=.6]{latex_page_garde/udbl.png}
    \end{center}

    \begin{center}
        \rule{\linewidth}{1pt}
    \end{center}
    \begin{center}
        \textbf{\large #1 \\
        (\emph{cas de MMG et de l'ARPTC})
        }
    \end{center}
    \begin{center}
        \rule{\linewidth}{1pt}
    \end{center}

    \vspace{5pt}

    \begin{FlushRight}
        \emph{Travail présenté et défendu en vue de l’obtention  \\  du grade d’ingénieur technicien en Informatique} \\
        \vspace{5pt}
        Présenté par :\textbf{#2} \\
        Filière : \textbf{#5} \\
    \end{FlushRight}
    \vspace{5pt}
    \begin{center}
        \textbf{\large Août 2025}
    \end{center}
    \newpage

    % =============================================== PAGE II ===================================================================

    \begin{center}
        UNIVERSIT\'E DON BOSCO DE LUBUMBASHI \\
        Faculté des Sciences Informatiques \\
        Département Réseaux\\
        Lubumbashi \\
        www.udbl.ac.cd \\
    \end{center}

    \begin{center}
        \rule{\linewidth}{2pt}
    \end{center}
    % ++++++++++++++++++++++++++++++ LE LOGO + LE TITRE
    \vspace{10pt}
    \begin{center}
        \includegraphics[scale=.6]{latex_page_garde/udbl.png}
    \end{center}

    \begin{center}
        \rule{\linewidth}{1pt}
    \end{center}
    \begin{center}
        \textbf{\large #1 \\
        (\emph{cas de MMG et de l'ARPTC})
        }
    \end{center}
    \begin{center}
        \rule{\linewidth}{1pt}
    \end{center}

    \vspace{5pt}

    \begin{FlushRight}
        \emph{Travail présenté et défendu en vue de l’obtention \\ du grade d’ingénieur technicien en Informatique} \\
        \vspace{5pt}
        Présenté par : \textbf{#2} \\
        Filière : \textbf{#5} \\
        \vspace{5pt}
        Directeur : \textbf{#3} \\
        Co-directeur : \textbf{#4} \\
    \end{FlushRight}
    \vspace{5pt}
    \begin{center}
        \textbf{\large Août 2025}
    \end{center}
}
\newpage

%         \begin{document}
%            \pagedegardeudbl{Migration de la telephonie classique vers la ToIP}
%                            {André MUTOKE}
%                            {Emile MUTOKE}
%                            {Baudouin BANZA}
%                            {L4 TLC et Reseaux}
%         \end{document}
\newcommand{\pagedegardeudbl}[5]{
    \pagenumbering{gobble}
    \begin{center}
        UNIVERSIT\'E DON BOSCO DE LUBUMBASHI \\
        Faculté des Sciences Informatiques \\
        Département Réseaux\\
        Lubumbashi \\
        www.udbl.ac.cd \\
    \end{center}
    \begin{center}
        \rule{\linewidth}{2pt}
    \end{center}

    \vspace{10pt}
    \begin{center}
        \includegraphics[scale=.6]{latex_page_garde/udbl.png}
    \end{center}

    \begin{center}
        \rule{\linewidth}{1pt}
    \end{center}
    \begin{center}
        \textbf{\large #1 \\
        (\emph{cas de MMG et de l'ARPTC})
        }
    \end{center}
    \begin{center}
        \rule{\linewidth}{1pt}
    \end{center}

    \vspace{5pt}

    \begin{FlushRight}
        \emph{Travail présenté et défendu en vue de l’obtention  \\  du grade d’ingénieur technicien en Informatique} \\
        \vspace{5pt}
        Présenté par :\textbf{#2} \\
        Filière : \textbf{#5} \\
    \end{FlushRight}
    \vspace{5pt}
    \begin{center}
        \textbf{\large Août 2025}
    \end{center}
    \newpage

    % =============================================== PAGE II ===================================================================

    \begin{center}
        UNIVERSIT\'E DON BOSCO DE LUBUMBASHI \\
        Faculté des Sciences Informatiques \\
        Département Réseaux\\
        Lubumbashi \\
        www.udbl.ac.cd \\
    \end{center}

    \begin{center}
        \rule{\linewidth}{2pt}
    \end{center}
    % ++++++++++++++++++++++++++++++ LE LOGO + LE TITRE
    \vspace{10pt}
    \begin{center}
        \includegraphics[scale=.6]{latex_page_garde/udbl.png}
    \end{center}

    \begin{center}
        \rule{\linewidth}{1pt}
    \end{center}
    \begin{center}
        \textbf{\large #1 \\
        (\emph{cas de MMG et de l'ARPTC})
        }
    \end{center}
    \begin{center}
        \rule{\linewidth}{1pt}
    \end{center}

    \vspace{5pt}

    \begin{FlushRight}
        \emph{Travail présenté et défendu en vue de l’obtention \\ du grade d’ingénieur technicien en Informatique} \\
        \vspace{5pt}
        Présenté par : \textbf{#2} \\
        Filière : \textbf{#5} \\
        \vspace{5pt}
        Directeur : \textbf{#3} \\
        Co-directeur : \textbf{#4} \\
    \end{FlushRight}
    \vspace{5pt}
    \begin{center}
        \textbf{\large Août 2025}
    \end{center}
}
\newpage

%         \begin{document}
%            \pagedegardeudbl{Migration de la telephonie classique vers la ToIP}
%                            {André MUTOKE}
%                            {Emile MUTOKE}
%                            {Baudouin BANZA}
%                            {L4 TLC et Reseaux}
%         \end{document}
\newcommand{\pagedegardeudbl}[5]{
    \pagenumbering{gobble}
    \begin{center}
        UNIVERSIT\'E DON BOSCO DE LUBUMBASHI \\
        Faculté des Sciences Informatiques \\
        Département Réseaux\\
        Lubumbashi \\
        www.udbl.ac.cd \\
    \end{center}
    \begin{center}
        \rule{\linewidth}{2pt}
    \end{center}

    \vspace{10pt}
    \begin{center}
        \includegraphics[scale=.6]{latex_page_garde/udbl.png}
    \end{center}

    \begin{center}
        \rule{\linewidth}{1pt}
    \end{center}
    \begin{center}
        \textbf{\large #1 \\
        (\emph{cas de MMG et de l'ARPTC})
        }
    \end{center}
    \begin{center}
        \rule{\linewidth}{1pt}
    \end{center}

    \vspace{5pt}

    \begin{FlushRight}
        \emph{Travail présenté et défendu en vue de l’obtention  \\  du grade d’ingénieur technicien en Informatique} \\
        \vspace{5pt}
        Présenté par :\textbf{#2} \\
        Filière : \textbf{#5} \\
    \end{FlushRight}
    \vspace{5pt}
    \begin{center}
        \textbf{\large Août 2025}
    \end{center}
    \newpage

    % =============================================== PAGE II ===================================================================

    \begin{center}
        UNIVERSIT\'E DON BOSCO DE LUBUMBASHI \\
        Faculté des Sciences Informatiques \\
        Département Réseaux\\
        Lubumbashi \\
        www.udbl.ac.cd \\
    \end{center}

    \begin{center}
        \rule{\linewidth}{2pt}
    \end{center}
    % ++++++++++++++++++++++++++++++ LE LOGO + LE TITRE
    \vspace{10pt}
    \begin{center}
        \includegraphics[scale=.6]{latex_page_garde/udbl.png}
    \end{center}

    \begin{center}
        \rule{\linewidth}{1pt}
    \end{center}
    \begin{center}
        \textbf{\large #1 \\
        (\emph{cas de MMG et de l'ARPTC})
        }
    \end{center}
    \begin{center}
        \rule{\linewidth}{1pt}
    \end{center}

    \vspace{5pt}

    \begin{FlushRight}
        \emph{Travail présenté et défendu en vue de l’obtention \\ du grade d’ingénieur technicien en Informatique} \\
        \vspace{5pt}
        Présenté par : \textbf{#2} \\
        Filière : \textbf{#5} \\
        \vspace{5pt}
        Directeur : \textbf{#3} \\
        Co-directeur : \textbf{#4} \\
    \end{FlushRight}
    \vspace{5pt}
    \begin{center}
        \textbf{\large Août 2025}
    \end{center}
}
\newpage


\begin{document}
    %% ======================= INSERER LE TITRE DE VOTRE TRAVAIL ICI ================================
    \pagedegardeudbl{TITRE DE VOTRE TRAVAIL}
    {MUTOKE MUSULE André}
    {Msc Ir KAPULULA Dubois}
    {Msc Ir KABALA Larry}
    {Réseaux et Télécommunications}
    \newpage

    % ===================================================== Definition des entete et pied de page ===================================================

    % ============== Augmenter la hauteur de l'en-tête si nécessaire
    \setlength{\headheight}{13pt}
    \setlength{\headwidth}{15cm}

    \pagestyle{fancy}
    \lhead{}
    \chead{}
    \rhead{Page | \thepage}
    \renewcommand{\footrulewidth}{1pt}
    \lfoot{TFC-UDBL-TLC}
    \cfoot{}
    \rfoot{2025}
% ================================================== FIN DE LA DEFINITON DES ENTETES ET PIED DE PAGE ==============================================

    % Document proprement dit
    \cleardoublepage % Pour s'assurer que la numérotation reprend à la page suivante
    \pagenumbering{Roman}
    \setcounter{tocdepth}{3}
    \setcounter{chapter}{-1}

    

    \addcontentsline{toc}{chapter}{ÉPIGRAPHE}
    \chapter*{ÉPIGRAPHE}

    Le premier principe est que vous ne devez pas vous duper vous-même – 
    et vous êtes la personne la plus facile à duper. \\ \\ \\
    — Richard Feynman

    
    \cleardoublepage % Pour s'assurer que la numérotation reprend à la page suivante

    \addcontentsline{toc}{chapter}{DÉDICACE}
    \chapter*{DÉDICACE}
La dédicace d'un travail scientifique est une courte inscription placée au début de l'ouvrage 
(souvent après la table des matières) pour rendre hommage à une ou plusieurs personnes 
(famille, amis, mentors) qui ont apporté un soutien moral ou intellectuel crucial, tout en étant différente 
des remerciements plus formels qui gratifient l'ensemble des collaborateurs académiques et professionnels, 
marquant ainsi la dimension humaine derrière le projet scientifique.

En quoi consiste-t-elle ?
\begin{itemize}
    \item Un hommage personnel : C'est un espace pour exprimer gratitude et affection à ceux qui vous ont soutenu 
                                (parents, conjoint, amis) durant les longs mois de recherche, apportant ainsi une touche 
                                personnelle et chaleureuse à l'œuvre.
    \item Une inscription succincte : Elle doit être brève, souvent écrite en italique et placée en haut à droite de la page, 
                                selon des conventions de mise en page spécifiques.
    \item Un acte symbolique : Elle matérialise le lien entre l'auteur et son œuvre, attestant de sa présence et de 
                                son parcours, et se distingue des remerciements qui listent les contributions techniques et 
                                professionnelles. 
\end{itemize}

    \cleardoublepage % Pour s'assurer que la numérotation reprend à la page suivante

    \addcontentsline{toc}{chapter}{REMERCIEMENT}
    \chapter*{REMERCIEMENT}

Ce travail de fin de cycle n'aurait pu être mené à terme sans l'apport précieux et 
le soutien de nombreuses personnes et institutions. Nous tenons à exprimer notre 
profonde gratitude à :

L'Université Don Bosco de Lubumbashi, pour le cadre académique stimulant et les 
ressources mises à notre disposition, qui ont été essentiels à la réalisation de 
cette recherche.

Notre directeur, pour son encadrement rigoureux, ses 
conseils avisés et sa disponibilité constante, qui ont grandement contribué à la 
structuration et à la qualité de ce mémoire.

Notre co-directeur, pour son expertise technique, ses 
orientations pertinentes et son soutien indéfectible tout au long de ce projet.

Nos collègues et amis du département de Réseaux et Télécommunications, pour leur 
collaboration, leurs échanges enrichissants et leur encouragements qui ont
embelli cette aventure académique.
Que chacun trouve ici l'expression de notre sincère reconnaissance. \\ \\
\begin{FlushRight}
\textbf{L'auteur}
\end{FlushRight}
    \cleardoublepage % Pour s'assurer que la numérotation reprend à la page suivante

    \addcontentsline{toc}{chapter}{LISTE DES FIGURES}
    \listoffigures
    \cleardoublepage % Pour s'assurer que la numérotation reprend à la page suivante

    \addcontentsline{toc}{chapter}{LISTE DES TABLEAUX}
    \listoftables
    \cleardoublepage % Pour s'assurer que la numérotation reprend à la page suivante

    \addcontentsline{toc}{chapter}{LISTE D'ACRONYMES}
    \chapter*{LISTE D'ACRONYMES}

\begin{acronym}
    \acro{ARPTC}{Autorité de Régulation de la Poste et des Télécommunications au Congo}
    \acro{CE}{Conformité européenne}
    \acro{CEM}{Compatibilité électromagnétique}
    \acro{GUI}{Graphical User Interface}
    \acro{CMC}{China Minmetals Corporation}
    \acro{CMN}{China Minmetals Non-ferrous Metals Co. Ltd}
    \acro{ETSI}{European Telecommunications Standards Institute}
    \acro{CISPR}{Comité International Spécial des Perturbations Radioélectriques}
    \acro{FCC}{Federal Communications Commission}
    \acro{GPS}{Global Positioning System}
    \acro{IEC}{International Electrotechnical Commission}
    \acro{IEEE}{Institute of Electrical and Electronics Engineers}
    \acro{IHM}{Interface Homme Machine}
    \acro{InSAR}{Interferometric Synthetic Aperture Radar}
    \acro{LRU}{Least Recently Used}
    \acro{MBD}{Model Based Design}
    \acro{MMG}{Minerals and Metals Group}
    \acro{PABX}{Private Automatic Branch exchange}
    \acro{PIRE}{Puissance Isotrope Rayonnée Equivalente}
    \acro{PMR}{Professional Mobile Radio}
    \acro{RADAR}{RAdio Detection And Ranging}
    \acro{RDC}{République Démocratique du Congo}
    \acro{RF}{Radio Frequency}
    \acro{SAR}{Synthetic Aperture Radar}
    \acro{UIT}{Union internationale des télécommunications}
    \acro{VoIP}{Voice over Internet Protocol}
    \acro{HIRF}{High Intensity Radiated Field}
    \acro{RSSI}{Received Signal Strength Indicator}
    \acro{VNA}{Vectorial Network Analyser}
    \acro{FFT}{Fast Fourier Transform}
    \acro{VCO}{Voltage-Controlled Oscillator}
\end{acronym}

    \addcontentsline{toc}{chapter}{TABLE DES MATIÈRES}
    \tableofcontents
    \cleardoublepage % Pour s'assurer que la numérotation reprend à la page suivante

    \addcontentsline{toc}{chapter}{AVANT PROPOS}
    \chapter*{AVANT-PROPOS}
    \paragraph{}
    Le présent mémoire, intitulé « DÉVELOPPEMENT D’UN OUTIL D’AIDE À L’HOMOLOGATION DES SYSTÈMES RADARS EXPLOITÉS DANS LES SITES MINIERS », est le fruit de nos efforts pour répondre aux exigences du programme national de l'enseignement supérieur et universitaire en République Démocratique du Congo. Il est présenté en vue de l'obtention de la licence d'ingénieur technicien en Télécommunication et Réseaux à l'Université Don Bosco de Lubumbashi.

Ce travail s'inscrit dans une démarche visant à apporter une contribution significative et concrète à la sécurité des opérations minières, en proposant une solution aux défis de la conformité des systèmes radars. Il est le résultat de recherches approfondies, d'une analyse rigoureuse des besoins sur le terrain, et d'une volonté d'innover pour la protection des vies humaines et des biens. Nous espérons que les développements et les conclusions présentés ici serviront de base solide pour de futures avancées dans ce domaine crucial.


    \cleardoublepage % Pour s'assurer que la numérotation reprend à la page suivante
    \pagenumbering{arabic}
    \setcounter{page}{1}

    %================================================ CHAPITRE 0 ===============================================
    %================================================ CHAPITRE 0 ===============================================
\chapter{INTRODUCTION GÉNÉRALE}
\section{Contexte du travail}
    Etablir le contexte de votre travail de recherche est essentiel pour situer votre étude dans son environnement académique, 
    scientifique et pratique.
    Il s'agit de présenter les circonstances, les motivations et les enjeux qui ont conduit à la réalisation de ce travail.
    Le contexte doit inclure une revue succincte de la littérature existante, les lacunes
    identifiées, ainsi que les besoins spécifiques auxquels votre recherche vise à répondre.
    En définissant clairement le contexte, vous permettez au lecteur de comprendre l'importance et
    la pertinence de votre travail dans le domaine étudié.

\section{Problématique}
\paragraph{}
    La problématique d'un travail scientifique est une question centrale qui guide la recherche. Elle identifie un 
    problème spécifique ou un défi dans le domaine d'étude, en soulignant son importance et ses
    implications. La problématique doit être formulée de manière claire et concise, en mettant en évidence les
    lacunes dans les connaissances actuelles ou les besoins non satisfaits. Elle sert de fondement à la définition des objectifs 
    de la recherche et oriente la méthodologie employée pour y répondre.

\section{Hypothèse}
\paragraph{}
    Les hypothèses dans un travail scientifique sont des propositions ou des suppositions formulées avant la 
    réalisation de la recherche. Elles servent de points de départ pour l'investigation et guident
    la collecte et l'analyse des données. Les hypothèses doivent être claires, testables et basées sur une
    compréhension préalable du sujet. Elles permettent de structurer la recherche en définissant ce que l'on
    s'attend à découvrir ou à démontrer, et elles sont essentielles pour évaluer les résultats obtenus.
\section{Choix et intérêt du sujet}
\paragraph{}
    Justifier le choix et l'intérêt du sujet de votre travail scientifique est crucial pour démontrer sa pertinence et son impact potentiel.
    Il s'agit d'expliquer pourquoi ce sujet a été sélectionné, en mettant en avant son importance dans le domaine d'étude, les
    lacunes qu'il vise à combler, et les bénéfices attendus de la recherche
\section{Méthodologies et techniques}
    \subsection{Méthodologies}
    \paragraph{}
    Enumérer et décrire les méthodologies que vous avez adoptées pour mener à bien votre travail scientifique est essentiel pour assurer la rigueur et la crédibilité de votre recherche.
    Voici quelques méthodologies couramment utilisées dans les travaux scientifiques :
    \begin{itemize}
        \item \textbf{Méthode expérimentale} : Implique la manipulation de variables pour observer leurs effets, souvent utilisée dans les sciences naturelles et l'ingénierie.
        \item \textbf{Méthode descriptive} : Consiste à collecter des données pour décrire un phénomène ou une situation sans intervenir, couramment utilisée en sciences sociales.
        \item \textbf{Méthode comparative} : Compare différents groupes ou conditions pour identifier des différences ou des similitudes, utile dans diverses disciplines.
        \item \textbf{Méthode analytique} : Décompose un problème complexe en ses éléments constitutifs pour mieux le comprendre, souvent utilisée en mathématiques et en informatique.
        \item \textbf{Méthode qualitative} : Se concentre sur la compréhension des phénomènes à travers des données non numériques, telles que les entretiens et les observations.
        \item \textbf{Méthode quantitative} : Utilise des données numériques et des analyses statistiques pour tester des hypothèses et mesurer des variables.
        \item \textbf{Méthode mixte} : Combine des approches qualitatives et quantitatives pour bénéficier des avantages des deux méthodologies.
    \end{itemize}

    \subsection{Techniques}
    Enumerer et décrire les techniques spécifiques que vous avez utilisées pour collecter, analyser et interpréter les données dans votre travail scientifique est crucial pour assurer la transparence et la reproductibilité de votre recherche.
    Voici quelques techniques couramment employées dans les travaux scientifiques :
    \begin{itemize}
        \item \textbf{Enquêtes et questionnaires} : Utilisés pour collecter des données auprès d'un large échantillon de participants.
        \item \textbf{Entretiens} : Permettent d'obtenir des informations détaillées et qualitatives en interrogeant des individus ou des groupes.
        \item \textbf{Observation participante} : Implique l'immersion dans un environnement pour observer les comportements et les interactions.
        \item \textbf{Analyse statistique} : Utilisée pour traiter et interpréter des données quantitatives à l'aide de logiciels statistiques.
        \item \textbf{Modélisation informatique} : Permet de simuler des phénomènes complexes à l'aide de modèles mathématiques et informatiques.
        \item \textbf{Expérimentation en laboratoire} : Implique la réalisation d'expériences contrôlées pour tester des hypothèses spécifiques.
        \item \textbf{Analyse de contenu} : Technique qualitative utilisée pour analyser des documents, des textes ou des médias afin d'identifier des thèmes ou des patterns.
        \item \textbf{Techniques de visualisation des données} : Utilisées pour représenter graphiquement les données afin de faciliter leur interprétation.
    \end{itemize}

\section{État de l'art}
\paragraph{}
    Nous ne pouvons pas prétendre être le premier à faire des investigations dans ce domaine. Cepandant, UDBL nous recommande 
    de repertorié les travaux de fin d'étude qui ont été réalisés dans le même domaine que le vôtre. Ces travaux font reference
    aux étudiants qui vous ont précédé à UDBL/ESIS

    
\section{Délimitation du travail}
        Pour qu'un travail scientifique aboutisse et qu'il réponde à la demande de la société, il doit être projeté dans le temps et dans l'espace
    \begin{itemize}
        \item Limites temporelles : C'est le temps imparti pour la réalisation de ce travail, qui est généralement de six (6) mois, là
        que vous aurez planifié les différentes étapes de votre recherche, y compris la collecte de données, l'analyse, la conception, 
        l'implémentation et la rédaction du rapport final.
        \item Limite spatiale : C'est le lieu où se déroule notre étude. Par exemple, si votre travail porte sur l'homologation des systèmes radars dans les sites miniers,
        vous pouvez spécifier que votre étude se concentre sur une région géographique particulière ou sur un type spécifique de site minier.
    \end{itemize}

\section{Subdivision du travail}
    Subdiviser le travail scientifique en chapitres clairs et logiques est essentiel pour structurer la présentation de la recherche.
    
\section{Logiciels et équipements utilisés}
    Pour pouvoir élaborer notre travail, nous avons utilisé les outils suivants :
    \begin{itemize}
        \item \textbf{Latex} : Pour produire les documents annexes ainsi que les documents d'homo\-logation
        \item \textbf{Microsoft Office Excel} : pour produire les graphiques
        \item \textbf{Draw IO} : le logiciel de représentation nous permettant de faire des architectures et des organigrammes.
        \item \textbf{Visual Studio Code} : Un éditeur de texte.
    \end{itemize}

    %================================================ CHAPITRE 1 ===============================================
    %================================================ CHAPITRE 1 ===============================================

\chapter{ÉTUDE DE L'EXISTANT ET GÉNÉRALITÉ DES CONCEPTS}
\section{Introduction partielle}
\paragraph{}
Ce chapitre inaugural plonge au cœur de notre étude en dressant un état des lieux 
complet des systèmes existants et en posant les bases 
conceptuelles essentielles à la compréhension de notre travail

Nous débuterons par une présentation détaillée de l’entreprise, etc.

\section{Étude de l’existant \& Généralité des concepts}
\subsection{Présentation de l’entreprise XXXXX}
        \subsubsection{Cadre historique}
        \paragraph{}
        Une présentation d'un point de vue historique de l'entreprise(ou des entreprises).
        \subsubsection{Cadre géographique}
        \paragraph{}
        Une présentation d'un point de vue géographique de l'entreprise(ou des entreprises).
        \subsubsection{Cadre organisationnel}
        \paragraph{}
        Une présentation d'un point de vue organisationnel de l'entreprise(ou des entreprises).

\subsection{Étude de l'existant}
    L'étude de l'existant dans un travail scientifique consiste à faire un état des lieux détaillé du domaine d'étude 
    avant d'entreprendre une nouvelle recherche, afin de comprendre le contexte, identifier les forces et faiblesses du 
    système actuel, recenser les objectifs existants et les contraintes, et ainsi délimiter clairement le champ de 
    l'investigation pour définir les besoins et éviter les doublons, utilisant des méthodes comme l'interview ou 
    l'enquête.

\subsection{Critique de l'existant}
\paragraph{}
    \subsubsection{Points forts}
    \paragraph{}
    Les points forts du système existants.
    

    \subsubsection{Points faibles}
    \paragraph{}
    Les points faibles du système existants.
\subsection{Besoins fonctionnels \& non fonctionnels}
\subsubsection{Besoins fonctionnels}
\paragraph{}
Les besoins dont vous avez trouvez fonctionnels au sein de l'entreprise
\subsubsection{Besoins non fonctionnels}
\paragraph{}
Les besoins dont vous avez trouvez non fonctionnels au sein de l'entreprise.

\section{Conclusion partielle}
Une conclusion partielle de ce chapitre, suivie d'une transition vers le chapitre suivant.

    %================================================ CHAPITRE 2 ===============================================
        %================================================ CHAPITRE 2 ===============================================
\chapter{CONCEPTION DE L’OUTIL}
\section{Introduction partielle}

Ce chapitre se consacre à la conception détaillée de notre outil d'aide à l'homologation, s'appuyant sur les besoins et les critiques 
identifiés dans l'étude de l'existant. Nous y décrirons les méthodologies de conception adoptées, notamment le modèle en V et l'approche Model 
Based Design (MBD), avant de traduire les exigences en une architecture logique et fonctionnelle, incluant les diagrammes d'exigences, de cas 
d'utilisation, d'activités et l'interface homme-machine qui sera notre \ac{GUI}.
    
\section{Conception générale}
\paragraph{}
Dans un processus classique de conception des systèmes ; entre 
autres le processus \ac{LRU}; les ingénieurs recueillent des 
spécifications à partir de plusieurs sources et les combinent 
afin d'élaborer un cahier des charges sur papier, qui aidera  
à produire une 
conception détaillée. Une série des concepts différents seront 
analysées par le biais de prototypes des simulateurs ou des 
maquettes,
la conformité à ces spécifications sera aussi contrôlée, 
apportant ainsi les modifications appropriées\cite{lru_mdel}. 
Après avoir mis 
au point une conception acceptable, les tests 
de vérification et de validation seront exécutés, puis de 
conformité si le produit doit respecter certaines normes 
spécifiques\cite{lru_mdel}. 
           
Comme les tests se déroulent à la fin d’un processus en plusieurs 
étapes faisant appel à des équipes différentes, les erreurs 
introduites en 
phase de conception ne sont souvent détectées que bien plus 
tard\cite{lru_mdel}. 

Leur correction onéreuse oblige la direction à prendre 
des décisions budgétaires 
difficiles\cite{lru_mdel}.  

Pour tenir compte de la contrainte temporel, nous allons utiliser 
le modèle de conception en V et le modèle de conception \ac{MBD}. 

    \subsection{Le V-Shape Model}
            \paragraph{}
        Le cycle en V (V model ou  V-Shape model  en anglais) est un modèle d'organisation des activités d'un projet qui se caractérise par un flux 
        d'activité descendant qui détaille le produit jusqu'à sa réalisation, et un flux ascendant, qui assemble le produit en vérifiant sa qualité. 
        Ce modèle est issu du modèle en cascade dont il reprend l'approche séquentielle et linéaire de phases\cite{model_v}.

    \subsection{L'approche \ac{MBD}}
            \paragraph{}
        Le Model-Based Design (MBD) est une méthode de gestion de projet qui permet d'améliorer le développement d'un système. Alors qu'un développement 
        classique est basé sur le cycle en V, l'approche MBD consiste à modéliser le système au plus tôt pour tester les choix stratégiques \& 
        techniques (Check) dès le design du système (\emph{Plan}) tout en structurant et en enrichissant la communication entre les différentes équipes 
        chargées du développement. Le coût de la modélisation est compensé par le gain en temps de développement, la convergence plus rapide de la 
        solution et la réutilisation des modèles créés (parfois d'un facteur 101)\cite{model_based_definition}.

    \subsection{Analyse des besoins}
    \paragraph{}
        L'analyse des besoins se traduit par la réalisation du cahier de charge fonctionnel. Un cahier des charges fonctionnel est un document rassemblant 
        l'ensemble des éléments liés à un projet. Le besoin, les prestations attendues et les objectifs y sont détaillés, ainsi que les différentes 
        contraintes (techniques, juridiques ou budgétaires)\cite{SysML}.

        Dans le but de développer un outil qui aidera à réaliser l'homologation des équipements de télécommunications, plus particulièrement les équipements 
    RADAR utilisés par les entreprises minières, tout en se focalisant sur le point faible de l'évolution des technologies comme point faible à améliorer, 
    nous devons faire ressortir les différents besoins de notre système.
        \subsubsection{Définition des besoins fonctionnels}
        \paragraph{}
            Les besoins fonctionnels sont les exigences essentielles qui décrivent ce que le système doit faire, comme les fonctionnalités ou service à 
            fournir. Après études, nous pouvons énumérer les besoins suivants
            \begin{enumerate}
                \item Le système doit tester les performances techniques :Ici, il sera question de :
                \begin{itemize}
                    \item  la portée, précision, résolution angulaire, vitesse de rafraîchissement, etc.
                    \item  Immunité aux brouillages et aux interférences.
                    \item  Fiabilité et disponibilité.
                \end{itemize}

                \item Le système doit offrir la possibilité de soumettre des demandes d'homologation en ligne.
                \item Le système doit permettre la gestion des documents nécessaires à l'homologation
                      
                \item Le système doit tester la sécurité : ici, il s’agira de :
                \begin{itemize}
                    \item Conformité aux normes de sécurité (émissions électromagnétiques, pro\-tection contre les interférences, etc.).
                    \item Absence de risques pour les personnes et les biens.
                \end{itemize}

                \item Le système doit vérifier la \ac{CEM} : ici, il s’agira de veiller au respect des normes CEM pour éviter 
                les interférences avec d'autres systèmes. Pour notre travail, nous avons 
                opté pour la norme \ac{CISPR}, plus précisément la norme CISPR 25.
                

                \item Le système doit s'assurer de la sécurité de fonctionnement : ici, il s’agira de veiller aux mécanismes de sécurité pour prévenir les accidents et 
                les dysfonction\-nements du système.


            \end{enumerate}
            \subsubsection{Définition des besoins non fonctionnels}
            \paragraph{}
            Les besoins non fonctionnels concernent les critères de performance, la sécurité ou l'ergonomie du système\cite{SysML}. Nous pouvons citer :
                \begin{itemize}
                    \item Le système doit être conforme aux normes de sécurité des données (Norme IEEE)
                    \item Le système doit comporter une interface intuitive et accessible
                    \item Les tests devront se réaliser dans une chambre anéchoïque.
                \end{itemize}
            % \subsection{Difficultés d'homologuer les équipements telecoms }
            %     \begin{itemize}
            %         \item \textbf{Manque d'un milieu adequat pour effectuer les mesures} : Effectivement l'ARPTC 
            %     \end{itemize}

    \subsection{Analyse des contraintes à l'homologation des équipements RA\-DAR pour entreprises minière}
    \paragraph{}
    Les contraintes sont des fonctions qui impliquent des limitations quant à la conception d'un système. De ce fait, l'analyse des contraintes consiste à identifier
    les limitations ou restrictions qui pourraient affecter le projet. Nous devons prendre en compte ces contraintes pour mieux planifier le travail.
    \subsubsection{Contrainte technique} 
        Les contraintes techniques sont les contraintes liées à la technologie. Il nous faudra s'assurer de la compatibilité entre les parties 
        électroniques, électriques et informati\-ques du système.

    \subsubsection{Contraintes juridiques et politique}
    \paragraph{}
        Les contraintes juridiques sont les restrictions imposées par l'État congolais. Pour un bon fonctionnement, il faut s’acquitter des droits et des devoirs 
        auprès de l’autorité de régulation des postes et des télécommunications du Congo ARPTC et du ministère des télécommunications et des nouvelles 
        technologies de l’information et de la com\-munication PT\&NTIC\cite{travail_rdc}. 
        Les droits et les obligations, c’est le fait de s’acquitter en payant les titres 
        d’exploitation et être en ordre fiscal.

    \subsubsection{Contrainte Budgétaire}
    \paragraph{}
        Tout projet, aussi grand soit-il, possède des limites financière. Les contraintes budgé\-taires dictent combien on peut dépenser.
    % \subsection{Contraintes temporelles}

    % \subsection{Contrainte due aux ressources humaines}

    \subsubsection{Contrainte d'exploitation}
    \paragraph{}
        Les contraintes d'exploitation sont liées à l'utilisation du système. Elles englobent la maintenance, la fiabilité et la facilité d'utilisation du système.

    \subsection{Diagramme des exigences du système}
    \paragraph{}
        Une fois les exigences énumérées, le diagramme des exigences présenté dans la figure \ref{fig:figure3} vient résumer ce qui a été dit ci-haut.
        \begin{figure}[ht]
            \centering
            \includegraphics[scale=0.34]{mesDiagrammes/exigences/DiagrammeExigence}
            \caption{Diagramme des exigences}
            \label{fig:figure3}
        \end{figure}
    % \subsection{Définitions des activités générales de notre futur système}
    \section{Conception détaillée}
    \subsection{Conception logique du système : Cas d'utilisation}
\paragraph{}
    Après avoir conçu le diagramme des exigences résumant les spécifications fonctionnelles du système, nous allons maintenant passer à la 
    conception logique. Ici, il sera question de définir premièrement les acteurs intervenant dans le système. Un diagramme de cas d'utilisation 
    capture le comportement d'un système, d'un sous-système, tel qu'un utilisateur extérieur le voit. Il scinde la fonctionnalité du système en 
    unités cohérentes, les cas d'utilisation, ayant un sens pour les acteurs. 
    
    Les cas d'utilisation permettent d'exprimer le besoin des utilisateurs d'un système, ils sont donc une vision orientée utilisateur de ce 
    besoin au contraire d'une vision informatique.
    
    \subsubsection{Acteurs}
\paragraph{}
    Pour notre système, nous aurons comme acteurs :
    \begin{itemize}
        \item Utilisateur : L'utilisateur est un humain, qui va solliciter les différentes fonctionnalités du système. L'utilisateur généralise deux sous-catégories :
        \begin{itemize}
            \item L'autorité de Régulation
            \item L'entreprise cherchant homologation de ses équipements RADAR
        \end{itemize}
        \item Le système lui-même : Il intègre les cas d'utilisations favorisant l'interaction action avec l'utilisateur
        \item L'équipement à homologuer.
    \end{itemize}

    \subsubsection{Cas d'utilisations}
    \paragraph{}
    Un cas d'utilisation décrit une fonction qu'un système exécute pour atteindre l'objectif de l'utilisateur. Un cas d'utilisation doit renvoyer un résultat observable qui est utile pour l'utilisateur du système.

    Il s'agira principalement pour notre système de :
    \begin{itemize}
        \item Homologuer l'équipement : L'homologation est la compilation générale survenue lors des différents tests de conformité
            \begin{itemize}
                \item Objectif : Réaliser l'homologation de l'équipement RADAR
                \item Acteur principal : L'utilisateur
                \item Acteur secondaire :  Le système, l'équipement RADAR
                \item Préconditions : \begin{itemize}
                                        \item Avoir au préalable soumis une demande d'homologation
                                        \item Avoir Placer l'équipement à homologuer dans une chambre anéchoïque
                                    \end{itemize}
                \item Scénario : \begin{itemize}
                                    \item Ouvrir l'application
                                    \item Lancer l'homologation
                                    \item Insérer les informations
                                \end{itemize}
                \item Alternatif : Ne pas homologuer
                \item Exception : Ne pas homologuer si la base des données est saturée.
                \item Postcondition : Générer le document d'homologation.
            \end{itemize}
        \item Tester la conformité Radioélectrique : Tout en se basant sur l'arrêté ministériel ainsi que sur les réglementations techniques 
        spécifiées par l'UIT et IEC, il sera question ici de vérifier les performances techniques du RADAR testé. En d'autres termes, il 
        s'agit de s'assurer que l'appareil :
            \begin{itemize}
                \item Utilise les fréquences autorisées et dans les limites de puissance définies.
                \item N'interfère pas avec d'autres services radio
            \end{itemize}
            \begin{itemize}
                \item Objectif : Tester la conformité Radioélectrique
                \item Acteur principal : L'utilisateur
                \item Acteur secondaire :  Le système, l'équipement RADAR
                \item Préconditions : \begin{itemize}
                                        % \item Avoir au prealable soumis une demande d'homologation
                                        \item Avoir Placer l'équipement à homologuer dans une chambre anéchoïque
                                    \end{itemize}
                \item Scénario : \begin{itemize}
                                    \item Ouvrir l'application
                                    \item Lancer l'homologation ou tester les performances radio
                                    \item Tester la PIRE
                                    \item Tester la fréquence
                                    \item Tester la portée
                                    \item Tester le diagramme de rayonnement
                                \end{itemize}
                \item Alternatif : Ne pas réaliser le teste
                \item Exception : Réessayer plus tard.
                \item Postcondition : Passer au test de compatibilité, générer un rapport du test.
            \end{itemize}

        \item Tester la compatibilité Électromagnétique : Ce dernier vise à vérifier qu'un appareil électronique ne perturbe pas son environnement électromagnétique et qu'il peut fonctionner correctement dans cet environnement sans subir d'interférences.
        \begin{itemize}
            \item Objectif : Tester la compatibilité électromagnétique
            \item Acteur principal : L'utilisateur
            \item Acteur secondaire :  Le système, l'équipement RADAR
            \item Préconditions : \begin{itemize}
                                    % \item Avoir au prealable soumis une demande d'homologation
                                    \item Avoir Placer l'équipement à homologuer dans une chambre anéchoïque
                                \end{itemize}
            \item Scénario : \begin{itemize}
                                \item Ouvrir l'application
                                \item Lancer l'homologation ou tester les performances radio
                                \item Tester la norme IEC 
                                \item Tester la performance sur terrain.
                            \end{itemize}
            \item Alternatif : Ne pas réaliser le teste
            \item Exception : Réessayer plus tard.
            \item Postcondition : Passer au test de sécurité électrique et immunité sanitaire ;  générer un rapport test.
        \end{itemize}
        \item Tester la sécurité électrique et immunité sanitaire : visent à s'assurer que ces appareils ne présentent aucun danger pour l'utilisateur ou son environnement. Ils sont essentiels pour garantir la conformité aux normes de sécurité et protéger les personnes contre les risques d'électrocution, d'incendie ou d'autres accidents liés à l'électricité.
        \begin{itemize}
            \item Objectif : Tester la compatibilité électromagnétique
            \item Acteur principal : L'utilisateur
            \item Acteur secondaire :  Le système, l'équipement RADAR
            \item Préconditions : \begin{itemize}
                                    % \item Avoir au prealable soumis une demande d'homologation
                                    \item Avoir Placer l'équipement à homologuer dans une chambre anéchoïque
                                \end{itemize}
            \item Scénario : \begin{itemize}
                                \item Ouvrir l'application
                                \item Lancer l'homologation ou tester les performances radio
                                \item Tester la norme IEC 
                                \item Tester la performance sur terrain.
                            \end{itemize}
            \item Alternatif : Ne pas réaliser le teste
            \item Exception : Réessayer plus tard.
            \item Postcondition : Passer au test de sécurité électrique et immunité sanitaire ;   générer un rapport teste.
        \end{itemize}
        \item Soumettre demande d'homologation : L'utilisateur, autre que l'autorité de Régulation, devra avoir la possibilité de soumettre la 
        demande d'homologation et de l'envoyer à qui de droit.
        \begin{itemize}
            \item Objectif : Soumettre demande d'homologation
            \item Acteur principal : L'utilisateur
            \item Acteur secondaire :  Le système
            \item Préconditions : ---
            \item Scénario : Envoyer le mail
            \item Alternatif : Ne pas soumettre
            \item Exception : Réessayer plus tard.
            \item Postcondition : En attente d'une réponse.
        \end{itemize}
        
        \item Générer Document d'homologation : Une fois les différents testes réalisées, un document devra être généré. 
        \begin{itemize}
            \item Objectif : Générer le document de l'homologation
            \item Acteur principal : L'utilisateur
            \item Acteur secondaire :  Le système
            \item Préconditions : Avoir déjà réalisé l'homologation
            \item Scénario : Générer document
            \item Alternatif : Ne pas générer
            \item Exception : Réessayer plus tard.
            \item Postcondition : ---
        \end{itemize}
        
    \end{itemize}

    \subsubsection{Diagramme des cas d'utilisations}
    Une fois les acteurs ainsi que les cas d'utilisations recensés, nous pouvons résumer le tout à travers le diagramme des cas d'utilisation.
    \begin{figure}[ht]
        \centering
        \includegraphics[scale=0.34]{mesDiagrammes/userCase/Cas_d_utilisation}
        \caption{Diagramme des cas d'utilisation}
        \label{fig:figure4}
    \end{figure}

    \subsection{Conception logique du système : Activité}
    Le diagramme d'activité permet de vérifier la complétude d'un processus.
    Dans notre cas, nous allons nous focaliser sur le processus de demande d'homologation ainsi que le processus 
     général de l'homologation de nos équipements radar. 
    
    Suivant le flux de traitement du signal et de la communication, l'activité comprend :
    
    \begin{itemize}
        \item AMPLI LOG (Amplificateur Logarithmique) : C'est la première étape de la chaîne de réception du signal.
        Il reçoit le "Signal RF" (Radio Fréquence) de l'équipement à tester.
        Sa fonction est de "Déterminer la puissance du signal" en convertissant le signal RF en une tension proportionnelle au logarithme de la puissance d'entrée. Cela permet de mesurer une large plage de puissances.
        Le signal traité est ensuite transmis au "Downconverter".

        \item DOWNCONVERTER (Convertisseur Abaisseur de Fréquence) : Ce module est responsable de la "Transposition de fréquence intermédiaire".
        Il reçoit le signal de l'Ampli Log et un signal d'un "VCO" (Voltage-Controlled Oscillator - Oscillateur Commandé en Tension) pour effectuer un mélange de signaux.
        L'objectif est de ramener la fréquence du signal RF (qui peut être très élevée, comme les GHz pour les radars) à une "Fréquence intermédiaire" plus basse, plus facile à traiter par les composants numériques.
        Un "Filtre passe-bas" est appliqué après le mélange pour isoler la composante de basse fréquence (la fréquence intermédiaire) et éliminer les fréquences indésirables.


        \item VCO (Oscillateur Commandé en Tension) : Ce couloir représente le module qui "Génère une fréquence de 2.4 GHz".
        Cette fréquence est utilisée par le Downconverter comme fréquence locale pour le mélange.
        Le VCO est "Configuré via UART" (Universal Asynchronous Receiver/Transmitter), ce qui signifie que son fonctionnement (notamment la fréquence de sortie) est contrôlé par des commandes série venant de la partie logicielle.


        \item CONTRÔLEUR CENTRAL : Ce couloir gère la communication globale entre les différentes parties du système, particulièrement la communication série.
        Il est responsable de l'Initialisation de la communication UART (9600 bauds)". Cela établit le canal de communication entre le PC et les modules matériels.
        Il reçoit les "Données de mesure" des modules matériels (après conversion analogique-numérique).
        Il envoie des "Commandes UART" pour configurer et contrôler les modules matériels, comme le VCO.



        \item PC (Ordinateur Personnel) : C'est l'interface utilisateur et le centre de traitement des données.
        Le processus démarre par l'"Envoi de la commande de mesure vers les signaux" depuis le PC.
        Le PC "Récupère les données" via la communication UART.
        Il y a une décision "Fréquency mode" :
        \begin{itemize}
            \item  OUI : Si le mode fréquence est activé, une "FFT" (Fast Fourier Transform - Transformée de Fourier Rapide) est appliquée aux données pour analyser le spectre de fréquence du signal.
            \item NON : Si ce n'est pas le mode fréquence, le traitement continue.
        \end{itemize}
            Le PC est responsable de l'Affichage des signaux" sur une interface graphique, permettant à l'utilisateur de visualiser les résultats des mesures.
        Le processus se termine (cercle noir avec un contour rouge).
    \end{itemize}

L'utilisateur lance une commande de mesure depuis le PC.
Cette commande est transmise via UART pour initialiser la communication et configurer le VCO pour générer la fréquence nécessaire.
Le signal RF de l'équipement testé est d'abord amplifié de manière logarithmique pour mesurer sa puissance.
Ensuite, le signal est envoyé au Downconverter, qui utilise la fréquence du VCO pour transposer le signal RF vers une fréquence intermédiaire plus basse.
Les données de mesure sont ensuite envoyées du module de communication vers le PC.
Sur le PC, les données sont traitées : une FFT peut être appliquée si le mode fréquence est sélectionné, puis les résultats sont affichés graphiquement.
    \begin{figure}[ht]
        \centering
        \includegraphics[scale=0.4]{mesDiagrammes/activite/activite1.jpg}
        \caption{Diagramme d'activité du processus de demande d'homologation}
        \label{fig:figure5}
    \end{figure}

    La figure \ref{fig:figure5} nous présente le processus de demande d'homologation, tandis
que la figure \ref{fig:figure501}, nous présente le processus de mesure pour arriver
à homologuer les systèmes radar.

\begin{figure}[!h]
        \centering
        \includegraphics[scale=0.45]{mesDiagrammes/activite/ActivityHomologationmytest.png}
        \caption{Diagramme d'activité du processus d'homologation}
        \label{fig:figure501}
    \end{figure}

\subsection{Interface Homme Machine}
Partant de l'analyse des besoins, nous allons Réaliser un logiciel dont sa fonction principale est de faciliter l'homologation. De ce fait, ce dernier sera réaliser avec
le framework Qt. 
\begin{itemize}
    \item \textbf{Classes principales et leur rôle} : 
          \begin{enumerate}
            \item \textbf{KernelSystem} :
                  \begin{itemize}
                    \item Role : Gère la logique fondamentale de l'application, les modes de fonctionnement (Powermeter, Fréquence, Temporel) et l'interaction en mode 
                    console (bien que la partie console soit minimale dans le code actuel).
                    \item Membre clés : 
                        \begin{itemize}
                            \item $m\_myModeSys$ : MODE (énumération) :Mode de fonctionnement actuel.
                            \item $m\_enableConsolePrinting$: bool : Active/désactive l'impression console.
                            \item $m\_choice$: int : Choix de l'utilisateur en console.
                            \item $bufferSerialData$: std::stringstream* Buffer pour les données série (simplifié).
                        \end{itemize}
                    \item Méthodes Clés :
                        \begin{itemize}
                            \item KernelSystem(): Constructeur.
                            \item runningSystem(): Exécute la logique principale.
                            \item allowCLPrint(bool cmd): Active/désactive l'impression console.
                            \item changeMode(int mode): Change le mode de fonctionnement.
                            \item getMyMode(): Retourne le mode actuel.
                        \end{itemize}
                  \end{itemize}
            \item \textbf{Graphique} :
                  \begin{itemize}
                    \item Role : Gère l'affichage des différents types de graphiques (spectre, rayonnement, temporel) en utilisant QtCharts. 
                    C'est le composant visuel principal pour les données.
                    \item Hérite de: QChartView
                    \item Membre clés : 
                        \begin{itemize}
                            \item $m\_spectrumChart$: QChart* : Graphique cartésien pour le spectre.
                            \item $m\_spectrumSeries$: QLineSeries* : Série de données pour le spectre.
                            \item $m\_spectrumAxisX$, $m\_spectrumAxisY$: QValueAxis* : Axes pour le spectre.
                            \item $m\_radiationChart$: QPolarChart* : Graphique polaire pour le rayonnement.
                            \item $m\_radiationSeries$: QLineSeries* : Série de données pour le rayonnement.
                            \item $m\_radiationRadialAxis$, $m\_radiationAngularAxis$: QValueAxis*, QCategoryAxis* : Axes pour le rayonnement.
                            \item $m\_spectrumChartT$: QChart* : Graphique cartésien pour le mode temporel.
                            \item $m\_spectrumSeriesT$: QLineSeries* - Série de données pour le mode temporel.
                            \item $m\_spectrumAxisXT$, $m\_spectrumAxisYT$: QValueAxis* : Axes pour le mode temporel.
                            \item $myDataMap$: std::map<double, double> : (Note : à retirer, utilisé pour la simulation interne).
                            \item $m\_dataTimer$: QTimer* : (Note : à retirer, utilisé pour la simulation interne).
                            \item $m\_currentX$: double : (Note : à retirer, utilisé pour la simulation interne).
                        \end{itemize}
                    \item Méthodes Clés (Slots):
                        \begin{itemize}
                            \item updateSpectrumPlot(const QVector<QPointF> data): Met à jour le graphique de spectre.
                            \item updateRadiationPlot(const QVector<QPointF> data): Met à jour le graphique de rayonnement.
                            \item updateTemporalPlot(const QVector<QPointF> data): Met à jour le graphique temporel.
                            \item showSpectrumMode(): Active le mode spectre.
                            \item showRadiationMode(): Active le mode rayonnement.
                            \item showTemporalMode(): Active le mode temporel.
                        \end{itemize}
                    \item Méthodes Clés (Privées):
                        \begin{itemize}
                            \item initSpectrumChart(): Initialise le graphique de spectre.
                            \item initRadiationChart(): Initialise le graphique de rayonnement.
                            \item initTemporalChart(): Initialise le graphique temporel.
                        \end{itemize}
                    \item Signaux :
                        \begin{itemize}
                            \item signalToUpdateRadiationMode()
                            \item signalToUpdateSpectrumMode()
                            \item signalToUpdateTemporalMode()
                        \end{itemize}
                  \end{itemize}
            \item \textbf{BarreOutils} : 
                  \begin{itemize}
                    \item Role : Représente la barre d'outils principale de l'application, contenant des actions configurables.
                    \item Hérite de: QToolBar
                    \item Méthodes clés : 
                        \begin{itemize}
                            \item BarreOutils(const QString \&title, QWidget *parent = nullptr): Constructeur.
                        \end{itemize}
                    \item Méthodes Clés (Slots privés):
                        \begin{itemize}
                            \item onDisplayTriggered(), onMarkerTriggered(), onStimulusTriggered(), onCalibrateTriggered(), onRecallTriggered(), onConfigTriggered(), 
                            onPauseTriggered(): Gèrent les clics sur les actions et émettent des signaux.
                        \end{itemize}
                    \item Signaux :
                        \begin{itemize}
                            \item markerOptionTriggered()
                            \item displayOptionTriggered()
                            \item stimulusOptionTriggered()
                            \item calibrateOptionTriggered()
                            \item configOptionTriggered()
                        \end{itemize}
                  \end{itemize}
            \item \textbf{MainWindow} :
                  \begin{itemize}
                    \item Role : La fenêtre principale de l'application. Elle assemble tous les composants de l'interface utilisateur, 
                    gère les interactions entre eux et contient la logique de simulation des données.
                    \item Hérite de: QMainWindow
                    \item Membre clés : 
                        \begin{itemize}
                            \item $m\_myKernel$: KernelSystem* : Instance du système Kernel.
                            \item $dispalyButtonsContainer$, $markerButtonsContainer$, $stimulus\-ButtonsContainer$, $calibrateButtonsContainer$, 
                            $configButtons\-Container$: QWidget* : Conteneurs pour les boutons spécifiques à chaque section de la barre d'outils.
                            \item $centralWidget$: QWidget* : Widget central de la fenêtre.
                            \item $mainLayout$: QHBoxLayout* : Layout principal.
                            \item $myGraphic$: Graphique* : Instance du graphique.
                            \item $toolBar$: BarreOutils* : Instance de la barre d'outils.
                            \item $m\_menuBar$: QMenuBar* : Barre de menu.
                            \item $fileMenu$, $editMenu$, $viewMenu$, $runMenu$, $settingMenu$, $help\-Menu$: QMenu* : Menus individuels.
                            \item $m\_simulationTimer$: QTimer* : Timer pour la simulation de données.
                            \item $m\_displayTemporalFFTInSpectrumMode$: bool : Drapeau pour la FFT du temporel.
                        \end{itemize}
                    \item Méthodes Clés (Slots):
                        \begin{itemize}
                            \item updateSpectrumPlot(const QVector<QPointF> data): Met à jour le graphique de spectre.
                            \item updateRadiationPlot(const QVector<QPointF> data): Met à jour le graphique de rayonnement.
                            \item updateTemporalPlot(const QVector<QPointF> data): Met à jour le graphique temporel.
                            \item showSpectrumMode(): Active le mode spectre.
                            \item showRadiationMode(): Active le mode rayonnement.
                            \item showTemporalMode(): Active le mode temporel.
                        \end{itemize}
                    \item Méthodes Clés (Privées):
                        \begin{itemize}
                            \item initSpectrumChart(): Initialise le graphique de spectre.
                            \item initRadiationChart(): Initialise le graphique de rayonnement.
                            \item initTemporalChart(): Initialise le graphique temporel.
                        \end{itemize}
                    \item Signaux :
                        \begin{itemize}
                            \item signalToUpdateRadiationMode()
                            \item signalToUpdateSpectrumMode()
                            \item signalToUpdateTemporalMode()
                        \end{itemize}
                  \end{itemize}
          \end{enumerate}
\end{itemize}

\section{Conclusion partielle}
Ce chapitre a structuré l'élaboration de notre solution en présentant une conception robuste et méthodique. Nous avons défini les besoins 
fonctionnels et non fonctionnels, analysé les contraintes techniques, juridiques, budgétaires et d'exploitation, et traduit ces éléments en des 
diagrammes UML clairs et précis (exigences, cas d'utilisation, activités). La proposition d'une interface homme-machine intuitive complète cette 
phase de conception, fournissant une feuille de route détaillée pour l'implémentation de notre système, étape que nous aborderons dans le prochain 
chapitre.





    %================================================ CHAPITRE 3 ===============================================
    % =============================================================================================================================================================================
% ===============================================================================================================================================================================
% %%%%%%%%%%%%%%%%%%%%%%%%%%%%%%%%%%%%%%%%%%%%%%%%%%%% CHAPITRE 3 %%%%%%%%%%%%%%%%%%%%%%%%%%%%%%%%%%%%%%%%%%%%%%%%%%%%%%%%%%%%%%%%%%%%%%%%%%%%%%%%%%%%%%%%%%%%%%%%%

\chapter{IMPLÉMENTATION DU SYSTÈME}
\section{Introduction partielle}
Ce chapitre détaille la mise en œuvre pratique de notre système d'aide à l'homologation des systèmes radars, en s'appuyant sur les 
spécifications et la conception élaborées précédemment. Nous y explorerons les choix matériels et logiciels fondamentaux, depuis la 
conception de la chambre anéchoïque jusqu'à la mise en place du circuit d'acquisition des signaux RF et le développement de l'interface 
logicielle, en passant par les simulations nécessaires à la validation de notre approche.

\section{Matériels d'essais}

Il est impératif, pour mieux réaliser notre homologation d'un 
équipement non répertorié, de faire usage des matériels suivants :
\begin{itemize}
    \item \textbf{Chambre anéchoïque} : Cette dernière sera d'une taille adéquate pour permettre de maintenir un champ uniforme de dimensions suffisantes par rapport 
    au matériel à l'équipement RADAR. Des absorbants supplémentaires doivent être utilisés pour atténuer les réflexions dans les chambres qui ne sont pas entièrement 
    revêtues de matériau absorbant\cite{iec6000}. 
    \item \textbf{Circuit de contrôle pour générer les signaux RF} : Afin de commander l'équipement qui sera mis dans la chambre anéchoïque\cite{iec6000}.
    \item \textbf{Des sondes de champ isotopique}  dont les amplificateurs de tête et l'opto\-électronique présentent une immunité correcte aux champs à mesurer, et une 
    liaison à fibre optique avec l'indicateur situé à l'extérieur de la chambre. Il est également possible d'utiliser une liaison correctement filtrée. En effet, ces sondes
    doivent aussi subir un étalonnage\cite{iec6000}.

    \item \textbf{Matériels associés} : pour enregistrer les niveaux de puissance nécessaires à la valeur du
    champ requis et pour contrôler la génération de ce signal pour les essais.
    Des précautions doivent être prises pour que les matériels auxiliaires présentent une
    immunité suffisante\cite{iec6000}.
    % L’Annexe I donne une méthode d’étalonnage des sondes de champ E.
\end{itemize}

\section{Description de l'installation}

\subsection{La chambre anéchoïque}
L'homologation des équipements RADAR, bien évidement sur la partie qui consiste à tester les performances techniques, définit en effet ; des méthodes d'essais
pour évaluer l'incidence des rayonnements électromagnétiques sur le matériel concerné\cite{absorbeurTSA}.

La plupart des matériaux électroniques sont, dans une certaine mesure, perturbée par les rayonnements électromagnétiques. Cela est dû au fait qu'il existe plusieurs sources
de champs électromagnétiques.

Afin de mieux effectuer nos tests, la norme IEC 6000-4-3 prévoit l'installation d'une chambre 
anéchoïque d'ondes radiofréquences\cite{iec6000}.

\subsection{Types de chambre anéchoïque}
Il existe trois types de chambre anéchoïque :
\begin{itemize}
    \item \textbf{Chambre d'essai de la compatibilité électromagnétique CEM} : Une chambre d'essai CEM est un environnement intérieur spécialisé conçu spécifiquement 
    pour mesurer avec précision les émissions électromagnétiques produites par les appareils électroniques. Ces chambres anéchoïques sont conçues pour 
    répondre aux normes et exigences CEM\cite{absorbeurTSA}.

    La compatibilité électromagnétique (CEM) est définie par la capacité d’un appareil électrique/électronique à fonctionner nominalement dans l’environnement 
    électromagnétique pour lequel il est conçu. Tout fabricant de matériel électronique doit qualifier ou faire qualifier ses matériaux préalablement à leur 
    mise sur le marché\cite{siepel_notitle_nodate}. 
    
    Diverses réglementations fixent les seuils à respecter par les 
    équipements électroniques. Les essais d’émission et d’immunité 
    rayonnés 
    s’effectuent dans ce type de chambre anéchoïque\cite{siepel_notitle_nodate}.

    Différents modèles de chambres, semi ou complètement anéchoïques, répondent aux exigences normatives :
    \begin{enumerate}
        \item Pour des tests de préqualification ou de qualification,
        \item Pour des tests en champ rayonné de forte puissance \ac{HIRF}.
    \end{enumerate}
    
    \item \textbf{Chambres anéchoïques pour les mesures d’antennes} : Pour caractériser une antenne (diagramme de rayonnement par exemple, de type Radio/Télécom, Radar …), 
    il faut des conditions de transmission reproduisant au mieux l’espace libre. En théorie, il s’agit d’un environnement de mesure en champ lointain, 
    sans réflexion ni pollution électromagnétique\cite{siepel_notitle_nodate}.

    Ces chambres anéchoïques procurent une \emph{zone tranquille} dans laquelle les conditions de mesures (champ électrique, homogénéité …) sont maîtrisées.

    \item \textbf{Minichambres anéchoïques} : Les minichambres anéchoïques permettent de 
    tester des petits dispositifs IoT, RF/micro-ondes, cartes électroniques … 
    Elles représentent la solution idéale pour les tests R\&D, 
    prototypes ou contrôle de production de ces produits\cite{siepel_notitle_nodate}. 
    Comme une chambre anéchoïque grandeur nature, 
    elles créent un environnement électromagnétique répétitif et 
    isolent l’équipement à tester.
    
\end{itemize}
L'homologation consiste en trois types de test : le test CEM, 
le test radioélectrique ainsi que le test de sécurité électrique.
Pour réaliser le test de compatibilité électromagnétique, nous 
devons concevoir une \textbf{chambre d'essai de la compatibilité 
électromagnétique CEM}. Pour réaliser 
le test radioélectrique, nous devons concevoir 
une \textbf{chambre anéchoïque pour les mesures d’antennes.}\cite{noauthor_specification_2017}

\subsection{Dimensionnement de la chambre anéchoïque}

Une chambre anéchoïque est constituée d’une cage de Faraday couverte d’absorbants électromagnétiques. Ces chambres sont essentielles pour réaliser des essais de compatibilité
électromagnétique et de mesures d’antennes en conditions de champ libre\cite{noauthor_specification_2017}.

La cage de Faraday isole le matériel sous tests des ondes électromagnétiques polluantes 
extérieures, les absorbants absorbent les ondes électromagnétiques et empêchent toute réverbération.

Que ce soit au niveau de la chambre d'essais CEM ou de celle pour les mesures d'antennes, il nous faudra définir les propriétés des matériaux devant constituer nos chambres.

\subsubsection{Définition des dièdres absorbants}
La matière absorbante des ondes devra avoir une forme adéquate pour 
pouvoir annuler au mieux les réverbérations d'ondes électromagnétiques.\cite{absorbeurTSA} 
De ce fait la forme la plus appropriée est celle qui se rapproche au dièdre, entre autres, la pyramide.
\begin{figure}[ht]
    \centering
    \includegraphics[scale=0.8]{mesDiagrammes/img/pyramidAbsorberOrginal.png}
    \caption{Forme de matériau pyramidal absorbant}
    \label{fig:figure6}
\end{figure}
Ce matériau possède les caractéristiques suivantes :
\begin{itemize}
    \item Permittivité relative : $1$
    \item Perméabilité relative : $1$
    \item Conductivité électrique : $0.5 S/m$
\end{itemize} 
\begin{figure}[ht]
    \centering
    \includegraphics[scale=0.8]{mesDiagrammes/img/pyramidISOsurface.png}
    \caption{Phénomène d'absorption d'onde électromagnétique}
    \label{fig:figure7}
\end{figure}
Le matériau nécessaire est un absorbeur RF Ohmique. Les paramètres clés d'un absorbeur RF sont les suivantes :
\begin{itemize}
    \item \textbf{La Fréquence} : la gamme de fréquences dans laquelle nous souhaitons effectuer des mesures. Dans notre cas, nous travaillons avec des 
                                    micro-ondes
    \item \textbf{La réflexivité} : La quantité du signal RF entrant qui sera réfléchie. Plus la réflexivité est faible, plus l'absorption est importante.
    \item \textbf{Les dimensions} : Afin d'adapter l'absorbeur à la surface de notre chambre anéchoïque.
    \item \textbf{La forme} : Dans notre cas, nous optons pour une forme pyramidale
    \item \textbf{Le type de matériau} : définit les paramètres de performance, de qualité et de réflectivité.
\end{itemize}

Pour sa capacité à être absorbant dans toutes les directions,  nous choisissons le model pyramidal 
\emph{Coating Microwave Polyurethane Pyramidal Absorbers(TSA-PI series)}
de l'entreprise TESTUPS qui possède les caractéristiques suivantes\cite{absorbeurTSA} :
\begin{itemize}
    \item Fréquence : $8OMhz$ à $40Ghz$; les équipements radars fonctionnent à $1GHz$, donc c'est passable.
    \item Matériaux : Mousse de polyuréthane; cette dernière possède les propriétés suivantes :
                        \begin{itemize}
                            \item Légèreté : Elle est très légère, ce qui facilite l'installation des absorbeurs dans les chambres anéchoïques.
                            \item Capacité d'absorption : Grâce à sa structure poreuse et à la possibilité d'y incorporer des charges carbonées 
                            (qui sont les vrais éléments absorbants les ondes électromagnétiques), la mousse de polyuréthane constitue un support 
                            idéal pour les absorbeurs. Les ondes RF pénètrent la structure de la mousse et sont dissipées sous forme de chaleur, 
                            réduisant ainsi les réflexions.
                            \item Facilité de formage : Elle peut être moulée facilement en formes pyramidales, essentielles pour la performance 
                            des absorbeurs RF.
                        \end{itemize}
    \item 
\end{itemize}

Pour notre cas nous avons choisit le model \emph{Coating Mocrowave Polyurethane Pyrmidal Absorbers TSA-200PI} qui possède les caractéristiques
suivantes\cite{absorbeurTSA} : 
    \begin{itemize}
        \item Matériaux : Polyuréthane
        \item Température : [$-50^{o} C , 80^{o} C$]
        \item Bandwidth : $500$MHz - $40$Ghz
        \item Hauteur : $6.3Kg/m^{2}$
        \item Poids : $200 mm$
        \item Réflectivité : [$-15$dB, $-28$dB]
        \item Size : $500$mm×$500$mm
        \item Couleur Standard : Bleu
        \item Espérance de vie : $> 10$ans
        \item RoHS : 2011/65/EU(EU)2015/863
    \end{itemize}

\subsubsection{chambre d'essai de la compatibilité électromagnétique CEM}
Il existe les chambres semi-anéchoïques et totalement anéchoïques. 
Dans notre cas, pour réaliser les tests de compatibilité RF, 
nous aurons besoins d'une chambre totalement anéchoïques.
Cette dernière possédera les dièdres absorbants TSA-PI. 
Une superficie de $7$x$7m^{2}$ sera amplement suffisante.
Pour déterminer la quantité de nos dièdres absorbant on féra :
    \begin{equation}
        N_{diedre} = \frac{S_{chambre}}{Size_{diedre}} * 2
    \end{equation}
    \begin{equation*}
        N_{diedre} = \frac{14m^{2}}{250000m^{2}} * 2 = 52
    \end{equation*}

En guise de marge d'erreur nous augmentons $10\%$. ce qui nous donne $58$ TSA-200PI( de $500$mm×$500$mm).
\begin{figure}[ht]
    \centering
    \includegraphics[scale=0.8]{mesDiagrammes/img/anechoicChamber.png}
    \caption{Chambre anéchoïque}
    \label{fig:figure8}
\end{figure}
\begin{figure}[ht]
    \centering
    \includegraphics[scale=0.8]{mesDiagrammes/img/anechoicChamber2.png}
    \caption{Chambre anéchoïques avec antenne biconnique}
    \label{fig:figure9}
\end{figure}

Ceci étant fait, nous pouvons implémenter un \ac{VNA}
dédié.

\subsection{Analyseur des réseaux vectoriels}

Un VNA est un appareil de mesure couramment utilisé pour\cite{chetouani_developpement_nodate} :
\begin{itemize}
    \item Caractériser des composants RF, des câbles et les antennes, 
    \item Caractériser des matériaux (solides et liquides), 
    \item Tester la conformité de dispositifs électroniques aux normes réglementaires, comme les tests CEM, 
    \item Caractériser les dispositifs électro-optique, opto-électrique ou optique, 
    \item Vérifier le déphasage en fonction de la fréquence : gain d’amplificateur sur une bande passante donnée, 
    \item Sur un câble, mesurer des paramètres de transmission, de diaphonie et de localiser les défauts, 
    \item Mesurer la directivité d’une antenne sur une bande de fréquence donnée, 
    \item Mesurer la transmission, le coefficient de réflexion, le gain, le diagramme de rayonnement, la bande passante et l’impédance tout au long des processus de conception et de production
\end{itemize}

Le circuit du VNA sera constitué :
\begin{itemize}
    \item Une partie logique : Un logiciel qui se chargera d’être une interface homme machine et fournir a son utilisateur différentes fonctionnalités
    \item Une partie physique (électronique) : Tout ce qui concernant le traitement des signaux, la récolte des données, sera réaliser par cette dernière. 
\end{itemize}

Nous nous baserons sur le schéma bloc de la 
figure \ref{fig:figure13} pour réaliser nos système de mesures RF afin de mieux homologuer nos systèmes RADAR

\begin{figure}[ht]
    \centering
    \includegraphics[scale=0.5]{mesDiagrammes/MesureRF2.png}
    \caption{Schéma bloc du système de mesure RF}
    \label{fig:figure13}
\end{figure}

\subsubsection{Récepteur des signaux RF}

Le système de réception des signaux RF est divisé en deux grande partie :

\begin{itemize}
    \item La réception en mode Puissance mètre : Ce mode de réception consiste à traiter le signal reçu et fournir un signal qui sera le logarithme du signal d'entré. Partant de ce signal
            nous pouvons déterminer le \ac{RSSI} du signal reçu.
    \item La réception en mode temporel : En mode temporel, ici nous recuperons le signal en bande RF, on le remet en bande de base pour au final être numérisé. et des traitement
            postérieur seront exécutés par la suite.
\end{itemize}

Avant de commencer tout processus, il est important de travailler en bande de base, pour 
les raisons suivantes :
\begin{itemize}
    \item les composants devants effectuer le traitement du signal sont limités en terme de 
fréquence de travail.
    \item Le microcontrôleur chargé de réaliser le prétraitement des données possède
          un fréquence limite échantillonnage (40MHz maximum).
\end{itemize} 
 Pour y parvenir, nous devons réaliser ce qu'on appel, \textbf{la transposition de fréquence}, 
en utilisant un mélangeur des signaux downconverter. 

\subsubsection{Transposition des fréquences}
La transposition de fréquence consiste à transposer un signal dont le spectre est 
centré sur une fréquence initiale vers une autre fréquence sans altération de la 
bande passante\cite{roodaki_lavasani_fard_radiation_2024}.

Pour ce faire un nous aurons besoins de élément ci dessous présentés dans la figure \ref{fig:figure90} :
\begin{itemize}
    \item Un oscillateur local
    \item Un multiplieur des signaux(ou comparateur de phase)
\end{itemize}

\begin{figure}[ht]
    \centering
    \includegraphics[scale=0.5]{mesDiagrammes/melangeurSignaux.png}
    \caption{Schéma bloc du mélangeur des signaux}
    \label{fig:figure90}
\end{figure}

Au niveau de l'oscillateur local, nous avons un signal $V_{LO}$ comme decrit l'équation
\ref{eq:equation6}

\begin{equation}
   V_{LO}(t) = |V_{LO}| sin(\Omega_{e} t + \Phi_{LO}(t)) 
    \label{eq:equation6}
\end{equation}
\begin{equation}
    V_{s}(t) = K * V_{e}(t) * V_{LO}(t)
    \label{eq:equation7}
\end{equation}

Ce qui nous mene a conclure que le signal $V_{s}(t)$ sera donné par l'équation \ref{eq:equation8}

\begin{equation}
    V_{s}(t) = \frac{K .|V_{e}(t)|.|V_{LO}(t)|}{2} \left[cos(2\Omega_{e} t + \Phi_{e}(t) + \Phi_{LO}(t)) + cos( \Phi_{e}(t) - \Phi_{LO}(t))\right]
    \label{eq:equation8}
\end{equation}

avec $\frac{K .|V_{e}(t)|.|V_{LO}(t)|}{2}$ la transmittance statique ou sensibilité
du comparateur de phase.

En associant ce circuit à un filtre passe bas, obtenons un mélangeur des signaux 
downconverter comme décrit la figure \ref{fig:figure901}

\begin{figure}[ht]
    \centering
    \includegraphics[scale=0.5]{mesDiagrammes/melangeurSignauxFiltrePassebas.png}
    \caption{Schéma bloc du mélangeur des signaux avec filtre passe bas}
    \label{fig:figure901}
\end{figure}

De manière pratique nous nous sommes proposer d'utiliser le module LTC5510(voir figure \ref{fig:figure903}). Les raisons
principale de notre choix sont les suivantes:
\begin{itemize}
    \item Fonction principale : Mélangeur actif large bande haute linéarité (Up-converter ou Down-converter).
    \item Plage de fréquences :
    \item Fréquence d'entrée/LO : 1 MHz à 6 GHz.
    \item Fréquence de sortie (FI) : Peut varier, par exemple, 10 MHz à 1,3 GHz pour des applications basse fréquence, ou 1,2 GHz à 2,1 GHz pour des applications haute fréquence.
    \item Gain de conversion : Typiquement 1,5 dB.
    \item Facteur de bruit (Noise Figure - NF) : Typiquement 11,6 dB (ou 11,8 dB selon les conditions de test).

    \item Niveau de pilotage de l'oscillateur local (LO Drive Level) : Nécessite seulement 0 dBm, ce qui simplifie les exigences du circuit de pilotage externe.
    \item Tension d'alimentation : 5V ou 3.3V.
\end{itemize}

\begin{figure}[ht]
    \centering
    \includegraphics[scale=0.6]{mesDiagrammes/shemaLTC5510.png}
    \caption{Schéma interne du LTC5510\cite{LT5510}}
    \label{fig:figure902}
\end{figure}

\begin{figure}[ht]
    \centering
    \includegraphics[scale=0.2]{mesDiagrammes/img/LTC5510.jpeg}
    \caption{Le module LTC5510}
    \label{fig:figure903}
\end{figure}

Pour réaliser le rôle de l'oscillateur local, nous nous sommes penché sur le circuit
ADF4351 (voir la figure \ref{fig:figure904}), qui est un synthétiseur des fréquence à larges bande avec boucle à verrouillage de phase
et possède les caractéristiques suivantes :
\begin{itemize}
    \item Gamme de fréquences: 35 MHz à 4.4 GHz. 
    \item Type de synthétiseur: Fractionnaire-N et entier-N. 
    \item VCO: Intégré, avec une fréquence de sortie fondamentale de 2200 MHz à 4400 MHz. 
    \item Prédiviseurs: Diviseurs par 1, 2, 4, 8, 16, 32 et 64 pour générer des fréquences plus basses. 
    \item Interface de contrôle: Bus série à trois fils. 
    \item Détecteur RMS: Détecteur RMS sélectif en fréquence. 
    \item Plage dynamique: 90 dB. 
\end{itemize}

\begin{figure}[ht]
    \centering
    \includegraphics[scale=0.1]{mesDiagrammes/img/ADF4351.jpeg}
    \caption{Le module ADF4351}
    \label{fig:figure904}
\end{figure}


\subsubsection{Réception en mode puissance-mètre}
L'outils de mesure doit être très précis et très fiable. Dans le cas de 
l'instrumentation pour l'homologation de nos système RADAR, il est nécessaire que la tension de 
sortie $V_{S}$ de nos amplificateur soit  une fonction concave du courant $i_{e}$; une phase 
étendue de niveaux du signal d'entrée et ainsi transformée en gamme compatible avec les possibilités
du système de numérisation et d'enregistrement\cite{roodaki_lavasani_fard_radiation_2024}.

\begin{figure}[ht]
    \centering
    \includegraphics[scale=0.4]{mesDiagrammes/AmpliLog3.png}
    \caption{Ampli logarithmique}
    \label{fig:figure10}
\end{figure}

La figure \ref{fig:figure10} possède les avantages d'un faible temps de réponse et d'une impédance
de sortie réduite.
\begin{itemize}
    \item $Rs$ : est la résistance d'isolement du capteur RF, ici dans notre cas de l'antenne.
    \item $Rf$ : est la résistance d'isolement de la chaîne directe de l'amplificateur
    \item $G$ : est le gain de l'amplificateur
    \item $i_{e}$ : représente le courant d'entrée de la chaîne directe
    \item $d_{0}$ : est la source de tension équivalente aux dérivées de la chaîne directe.
\end{itemize}

\begin{figure}[ht]
    \centering
    \includegraphics[scale=0.4]{mesDiagrammes/AmpliLog2Test.png}
    \caption{Étude de l'ampli logarithmique}
    \label{fig:figure101}
\end{figure}

Le fonctionnement en régime continue du circuit est alors décrit comme suite :
\begin{equation*}
    kV_{s} + V - d_{0} - \epsilon = 0
\end{equation*}
\begin{equation}
    \Rightarrow kV_{s} \left( 1 + \frac{1}{kG} \right)  - d_{0} + f \left(i_{e} - i_{0} + \frac{V_{s}}{G}\left[\frac{1}{R_{f}} + \frac{1}{R_{s}}\right] - d_{0}\left[\frac{1}{R_{f}} + \frac{1}{R_{s}}\right]\right) = 0
    \label{eq:equation1}
\end{equation}

Dans le cas idéal ou l'amplificateur de la chaîne directe est supposé sans dérivé, sans 
courant d'entrée et de Gain infini alors :
\begin{equation}
    kV_{s} \left(1 + \frac{1}{kG}\right)  - d_{0} + f\left(i_{e}\right) = 0
    \label{eq:equation2} 
\end{equation}
\begin{equation*}
    \Rightarrow kV_{s} + f\left(i_{e}\right) = 0    
\end{equation*}

\begin{equation}
    \Rightarrow V_{s} = \frac{1}{k} f\left(i_{e}\right)
    \label{eq:equation3}   
\end{equation}

En particulier, la caractéristique de l’élément de contre réaction est $V = a + b \log{i}$.
En pratique, cependant, la relation \ref{eq:equation2} souligne le rôle fondamental joué par
$i_{0}$, $d_{0}$ et $R_{f}$ même quand $G$ est infini. On aura en effet :
\begin{equation}
    kV_{s} - d_{0} = - f\left( i_{e} - i_{o} - d_{0}\left[\frac{1}{R_{s}} + \frac{1}{R_{f}} \right]\right)
    \label{eq:equation4}
\end{equation}

Dans l’équation \ref{eq:equation4}; on voit apparaître une erreur de lecture $d_{0}$ et 
une erreur de mesure $i_{0} + d_{0} \left[ \frac{1}{R_{s}} + \frac{1}{R_{f}} \right]$

Le circuit de contre réaction est simplement un élément à semi-conducteur. Dans notre exemple
un transistor.
\begin{equation*}
    I_{c} = I_{c_{1}} \left[e^{{V_{be}} / {U_{T}}} - 1 \right]
\end{equation*}
\begin{equation*}
    \Rightarrow \ln\left(\frac{I_{c}}{I_{c_{1}}} + 1 \right) = \frac{V_{be}}{U_{T}}
\end{equation*}

\begin{equation}
    \Rightarrow V_{be} =U_{T} \left[ \ln\left(I_{c}\right) - \ln\left(I_{c_{1}}\right)\right]
    \label{eq:equation5}
\end{equation}

\begin{figure}[ht]
    \centering
    \includegraphics[scale=0.4]{mesDiagrammes/AmpliLog2Transistor.png}
    \caption{Ampli logarithmique avec élément semi conducteur}
    \label{fig:figure11}
\end{figure}
Pour raison d'expérimentation, nous nous somme permis d'utiliser une carte de développement 
contenant le circuit de la figure \ref{fig:figure12}.
Le système RADAR rencontré fonctionnait dans la gamme  des micro-ondes. De ce fait nous 
disposons d'un amplificateur logarithmique AD8317 possédant les spécifications suivantes :
\begin{itemize}
    \item Bande de fréquence : $1$MHz - $10$GHz
    \item Haute précision : $\pm 1dB $ au dessus de la Temperature de travail.
    \item Stabilité au dessus de la Temperature de travail : $\pm 0.5 dB$
    \item Temps de réponse impulsionnelle : 6 ns/10 ns
    \item Impédance d'entrée : $55 \Omega$(avec une erreur inférieure à ± 3 dB)
    \item Tension d'alimentation : $3.0$ V à $5.5$ V 
    \item Consommation de courant : $22 mA$ et diminue à $200 \mu A$ lorsque l'appareil est désactivé
\end{itemize}
L'AD8317 est un amplificateur logarithmique démodulateur, capable de convertir avec précision un signal d'entrée RF en un signal de sortie correspondant, échelonné en décibels. 
Il utilise la technique de compression progressive sur une chaîne d'amplification en cascade, chaque étage étant équipé d'une cellule de détection.
 L'AD8317 présente un temps de réponse de 6 ns/10 ns  Ce dispositif offre une stabilité d'interception 
logarithmique sans précédent à température ambiante.
\cite{ad8317}.

\begin{figure}[ht]
    \centering
    \includegraphics[scale=0.6]{mesDiagrammes/img/ad8317_circuit.png}
    \caption{Circuit de l'amplificateur logarithme AD8317\cite{ad8317}}
    \label{fig:figure12}
\end{figure}

Le schéma sur la figure \ref{fig:figure131} nous présente les différentes connexions 
entre nos différents modules.

\begin{figure}[!ht]
    \centering
    \includegraphics[scale=0.7]{mesDiagrammes/myHOMOLOG.png}
    \caption{Circuit complet du système d'homologation}
    \label{fig:figure131}
\end{figure}

\subsection{Application de l'homologation (partie logiciel)}
Notre application sera implémenter au moyen du framework Qt Creator(voir figure
\ref{fig:figure14}), au
moyen du lange de programmation C++. Le code ci dessous décrit
les lignes de code  principaux du système.

\begin{lstlisting}[caption=Code principal, label=lst:moncode]
#include "mainwindow.h"

#include <QApplication>
#include <QIcon>


/// @brief  Classe principale du systeme
int main(int argc, char *argv[])
{
    QApplication a(argc, argv);
    // 1. Creer un objet QIcon a partir d'une ressource
    QIcon myIcon("F:/TFC_2025/mesDiagrammes/img/arptclogo2.png");
    MainWindow w;
    w.setWindowIcon(myIcon);
    w.show();
    return a.exec();
}
\end{lstlisting}

La fonction \ac{FFT} sera implémenter par notre application,  comme le décrit les
lignes de code qui suivent :
% \begin[inputencoding=utf8]{lstlisting}
%     QVector<QPointF> MainWindow::performFFT(const QVector<double>& timeDomainSamples, double samplingFrequencyHz)
% {
%     // Verifions le nombre des donnees recus
%     qDebug()<<"On veut faire la FFT\n";
%     int N = timeDomainSamples.size();
%     if (N == 0 || (N & (N - 1)) != 0) { // Vérifie si N est une puissance de 2
%         qWarning() << "FFT: Le nombre d'échantillons doit être une puissance de 2 et non nul.";
%         return {};
%     }
%     // Calculer log2(n)
%     int log2n = 0;
%     while ((1 << log2n) < N) {
%         log2n++;
%     }

%     // Structure pour les nombres complexes (partie réelle, partie imaginaire)
%     struct Complex {
%         double real;
%         double imag;
%         Complex(double r = 0.0, double i = 0.0) : real(r), imag(i) {}
%         Complex operator+(const Complex& other) const { return Complex(real + other.real, imag + other.imag); }
%         Complex operator-(const Complex& other) const { return Complex(real - other.real, imag - other.imag); }
%         Complex operator*(const Complex& other) const {
%             return Complex(real * other.real - imag * other.imag, real * other.imag + imag * other.real);
%         }
%     };

%     QVector<Complex> X(N); // Tableau pour les nombres complexes de la FFT

%     // 1. Initalisation avec les échantillons et conversion en complexes
%     for (int i = 0; i < N; ++i) {
%         qDebug()<<"On initialise le time domaine";
%         X[i] = Complex(timeDomainSamples[i], 0.0);
%     }
%     qDebug()<<"On finti avec le time domaine";

%     // 2. Réarrangement par inversion de bits (Bit-reversal permutation)
%     for (int i = 0; i < N; ++i) {
%         qDebug()<<"Reversal Bit";
%         int j = reverse_bits(i, log2n);
%         if (i < j) { // Éviter les échanges doubles
%             std::swap(X[i], X[j]);
%         }
%     }

%     // Étape 2 : Boucles itératives pour calculer la FFT/IFFT
%     // 'len' est la taille de la "fenêtre" ou du sous-problème courant
%     for (int len = 2; len <= N; len <<= 1) { // len = 2, 4, 8, ..., n
%         // 'ang' est l'angle pour les facteurs de rotation de cette étape
%         double ang = - 2 * M_PI / len;
%         // 'wlen' est le facteur de rotation de base pour cette longueur de fenêtre
%         Complex wlen(std::cos(ang), std::sin(ang));

%         // Parcourir les blocs de longueur 'len'
%         for (int i = 0; i < N; i += len) {
%             Complex w(1.0, 0.0); // 'w' est le facteur de rotation courant pour le bloc

%             // Appliquer l'opération papillon (butterfly operation) à l'intérieur de chaque bloc
%             // Les deux parties de la papillon sont séparées par len / 2
%             for (int j = 0; j < len / 2; j++) {
%                 // Les indices des éléments à combiner sont (i + j) et (i + j + len / 2)
%                 Complex u = X[i + j];
%                 Complex v = X[i + j + len / 2] * w; // Appliquer le facteur de rotation
%                 X[i + j] = u + v;
%                 X[i + j + len / 2] = u - v;
%                 w = w * wlen; // Mettre à jour le facteur de rotation pour la prochaine paire
%             }
%         }
%     }

%     // // 4. Conversion des résultats en magnitude (en dB) et fréquence (en MHz)
%     QVector<QPointF> fftResult;

%     // return magnitudes; // Retourne le vecteur des magnitudes du spectre fréquentiel.


%     // // 4. Conversion des résultats en magnitude (en dB) et fréquence (en MHz)
%     // QVector<QPointF> fftResult;
%     double maxMagnitude = 0.0; // Pour trouver le maximum pour la normalisation

%     // Nous ne prenons que la première moitié (N/2) des points de fréquence
%     // car la deuxième moitié est symétrique (pour les signaux réels).
%     // Les fréquences sont de 0 à la fréquence de Nyquist (Fs/2).
%     for (int i = 0; i < N / 2; ++i) {
%         // double magnitude_linear = std::sqrt(X[i].real * X[i].real + X[i].imag * X[i].imag);
%         double magnitude_linear = std::abs(X[i].real);
%         if (i == 0) { // Pour la composante DC (fréquence 0), la magnitude est déjà exacte
%             maxMagnitude = std::max(maxMagnitude, magnitude_linear);
%         } else {
%             // Pour les fréquences positives (i > 0), le spectre est double face.
%             // On multiplie par 2 pour obtenir la puissance totale à cette fréquence.
%             // (Ou alternativement, diviser par N et non par N/2 pour la magnitude absolue)
%             // Ici, on se concentre sur le niveau relatif.
%             maxMagnitude = std::max(maxMagnitude, magnitude_linear);
%         }
%     }
%     // Normaliser à 0 dB et convertir en dB
%     // Utiliser une petite valeur pour éviter log10(0)
%     const double MIN_DB_FLOOR = -100.0; // Plancher pour le graphique en dB
%     if (maxMagnitude == 0.0) maxMagnitude = 1e-9; // Éviter la division par zéro

%     for (int i = 0; i < N / 2; ++i) { // Traiter seulement jusqu'à la fréquence de Nyquist
%         double frequency_hz = (double)i * (samplingFrequencyHz / N); // Fréquence en Hertz
%         double magnitude_linear = std::sqrt(X[i].real * X[i].real + X[i].imag * X[i].imag);

%         double magnitude_db;
%         if (magnitude_linear > 1e-9 * maxMagnitude) { // Éviter log10(très petit nombre)
%             magnitude_db = 10 * std::log10(magnitude_linear / maxMagnitude);
%             // Si vous voulez la magnitude absolue, vous pouvez diviser par N plutôt que maxMagnitude
%             // Ou utiliser une référence de puissance (ex: 1Vrms = 0dBV)
%         } else {
%             magnitude_db = MIN_DB_FLOOR;
%         }
%         magnitude_db = std::max(magnitude_db, MIN_DB_FLOOR); // S'assurer du plancher minimum
%         fftResult.append(QPointF(frequency_hz, magnitude_db)); // Fréquence en MHz, amplitude en dB
%     }
%     qDebug() << "FFT calculée pour" << N << "points. Fréquence d'échantillonnage:" << samplingFrequencyHz << "Hz.";
%     return fftResult;
% }
% \end{lstlisting} 



\begin{figure}[ht]
    \centering
    \includegraphics[scale=0.21]{mesDiagrammes/QtCreator.png}
    \caption{Qt Creator}
    \label{fig:figure14}
\end{figure}

Par défaut lorsque l'application est lancé, c'est le mode puissance mètre
(voir figure \ref{fig:figure15}) qui est affiché.
L'utilisateur a la possibilité de changer des modes en selectionnant l'option paramètres offre à 
il s'affichera trois option :
\begin{itemize}
    \item Puissance mètre
    \item Domaine spectral
    \item Domaine temporel
\end{itemize}
\begin{figure}[!h]
    \centering
    \includegraphics[scale=0.4]{mesDiagrammes/pwrMeterMode.png}
    \caption{Application d'homologation}
    \label{fig:figure15}
\end{figure}
Pour établir une communication série avec les équipements de mesures, l'utilisateur
doit appuyer sur la barre d'outil \textbf{Config} et il devra configurer la communication
série et sélectionner le port de communication. Les options à spécifier sont :
\begin{itemize}
    \item Le port série,
    \item le débit en bauds (par défaut 9600 bauds),
    \item le nombre des bits des données(par défaut 8 bits),
    \item le bit de parité(par défaut aucun),
    \item le bit d’arrêt(par défaut le niveau logique 1),
    \item le contrôle de flux(Par défaut)
\end{itemize}

La communication série est établie via le protocole UART.Le protocole UART (Universal Asynchronous Receiver/Transmitter) se distingue par sa nature asynchrone. Contrairement aux protocoles synchrones qui partagent un signal d'horloge commun pour la synchronisation, l'UART s'appuie sur une synchronisation indépendante entre l'émetteur et le récepteur. Cette synchronisation est rendue possible grâce à des informations de cadencement contenues directement dans la trame de données.

Le débit de données ou Baud Rate est un paramètre critique qui définit la vitesse 
à laquelle les bits sont transmis. Il est exprimé en bits par seconde (bps). Afin 
que la communication soit établie, l'émetteur et le récepteur doivent 
impérativement être configurés avec le même débit. Des débits courants 
incluent 9600 bps, 19200 bps et 115200 bps. Si ce paramètre est mal configuré, 
le récepteur ne pourra pas interpréter correctement les bits reçus, ce qui 
entraînera une communication erronée.

Une communication UART est organisée en trames de données (ou Data Frames). 
Une trame est un paquet de bits qui encapsule les données utiles avec des bits 
de contrôle. La structure d'une trame standard est la suivante :
\begin{itemize}
    \item Bit de Start : Il s'agit d'un bit de valeur 0 qui marque le début de la trame. Il sert de signal au récepteur pour lancer sa propre horloge de synchronisation afin de lire les bits suivants.
    \item Bits de données : Ce sont les bits qui transportent l'information utile. Le nombre de bits de données est généralement de 8, mais il peut aussi être de 7.
    \item Bit de parité (optionnel) : Ce bit est utilisé pour la détection d'erreurs simples. Le bit est défini de manière à ce que le nombre total de bits à 1 dans la trame (y compris le bit de parité) soit pair (parité paire) ou impair (parité impaire).
 
    \item Bit(s) de Stop : Ce sont un ou deux bits de valeur 1 qui signalent la fin de la trame. Ils permettent de remettre la ligne à son état de repos.
\end{itemize}

La figure \ref{fig:figure16} une représentation schématique d'une trame de données UART :

\begin{figure}[!h]
    \centering
    \includegraphics[scale=0.4]{mesDiagrammes/trameUART.png}
    \caption{Trame UART}
    \label{fig:figure16}
\end{figure}

La ligne de communication, au repos, est maintenue à un niveau logique haut (1). 
La trame commence par un changement d'état vers le bas (0), ce qui active le récepteur.

En associant le tout, nous obtenons un analyseur des réseaux vectoriel (ARV) dédié, en 
anglais Vectorial Network Analyser(VNA) pour l'homologation
des systèmes RADAR.

\section{Conclusion partielle}
Ce chapitre détaille la mise en œuvre pratique de notre système d'aide à l'homologation des systèmes radars, en s'appuyant sur les 
spécifications et la conception élaborées précédemment. Nous y explorerons les choix matériels et logiciels fondamentaux, depuis la conception 
de la chambre anéchoïque jusqu'à la mise en place du circuit d'acquisition des signaux RF et le développement de l'interface logicielle, en 
passant par les simulations nécessaires à la validation de notre approche.

    
    %================================================ CHAPITRE 3 ===============================================
    % =============================================================================================================================================================================
% ===============================================================================================================================================================================
% %%%%%%%%%%%%%%%%%%%%%%%%%%%%%%%%%%%%%%%%%%%%%%%%%%%% CHAPITRE 4 %%%%%%%%%%%%%%%%%%%%%%%%%%%%%%%%%%%%%%%%%%%%%%%%%%%%%%%%%%%%%%%%%%%%%%%%%%%%%%%%%%%%%%%%%%%%%%%%%

\chapter{TESTS \& INTERPRÉTATION DES RÉSULTATS DU SYSTÈME}
\section{Introduction partielle}

Ce chapitre est dédié à la validation de notre système d'homologation à travers une série de tests rigoureux. Nous y présenterons les 
procédures de mesure RF effectuées au sein de la chambre anéchoïque, notamment l'acquisition des diagrammes de rayonnement et l'analyse de 
la compatibilité électromagnétique (CEM) conformément à la norme CISPR 25. L'interprétation des résultats obtenus permettra d'évaluer la 
performance et la conformité de l'équipement sous test, fournissant ainsi une validation concrète de l'outil développé.

\section{Test d'homologation}
Nous allons commencer par recupèrer des mesures RF. De ce fait, nous utiliserons 
le logiciel CST Studio comme simulateur d'Antenne. Dans notre cas de figure l'antenne
sera notre équipement sous test, en anglais \textbf{Équipment Under Test (EUT)}\cite{iec6000}.
L'antenne devra être placé dans la chambre anéchoïque et se verra générer un signal de 
test. Le signal de test est représenté dans la figure \ref{fig:figure42} 

\begin{figure}[ht]
    \centering
    \includegraphics[scale=0.5]{CST_Studio/TestingAntenna/testTingAntenna/Export/hornAntenna.png}
    \caption{Antenne en zone de test}
    \label{fig:figure41}
\end{figure}

Au sein du logiciel CST Studio, nous allons placer une antenne cornet(voir figure \ref{fig:figure41}), car le dispositifs
radar vu sur terrain était constitué de deux antennes cornets, un jouant le role
d’émetteur et l'autre jouant le rôle de récepteur.

Au tout debut, nous aurons à spécifier la fréquence de fonctionnement de l'antenne, 
comme nous spécifie la figure \ref{fig:figure411}.

\begin{figure}[ht]
    \centering
    \includegraphics[scale=0.5]{mesDiagrammes/frequenceAntenne.png}
    \caption{Fréquence de fonctionnement de l'antenne}
    \label{fig:figure411}
\end{figure}

\begin{figure}[ht]
    \centering
    \includegraphics[scale=0.5]{mesDiagrammes/img/signalTest.png}
    \caption{Signal test}
    \label{fig:figure42}
\end{figure}

La gamme de fréquence utilisée environne
les 10GHz(spécifique aux recommandation de l'UIT sur l'utilisation des ondes radio comme onde 
de détection radar), dans notre cas nous sommes entre 15GHz et 18GHz comme le spécifie
le tableau \ref{tab:table1}. Nous pouvons visualiser le signal en 3D dans CST Studio 
comme nous montre la figure \ref{fig:figure43}



\begin{figure}[!h]
    \centering
    \includegraphics[scale=0.6]{CST_Studio/TestingAntenna/testTingAntenna/Export/rayonnementDiag.png}
    \caption{Diagramme de rayonnement de l'EUT}
    \label{fig:figure43}
\end{figure}

Nous allons de ce fait récupérer les données du diagramme de rayonnement de
l'EUT dans un fichier et fournir ce dernier à notre logiciel d'homologation.

Ceci étant fait, nous lançons notre application et sélectionnons, à l'onglet
\emph{Paramètres}, l'option \emph{Puissance mètre}. Nous obtenons l'image
décrite par la figure \ref{fig:figure44}.

\begin{figure}[!h]
    \centering
    \includegraphics[scale=0.5]{mesDiagrammes/img/testRayonnement.png}
    \caption{Diagramme de rayonnement en mode Puissance mètre}
    \label{fig:figure44}
\end{figure}

 Avec le logiciel CST Studio, il est possible de réaliser le test 
 CEM, de ce fait nous irons dans l'onglet Post-processing, dans la barre d'outils, nous
 choisirons EMC Workflow dans l'option EMC Standard Limits. Nous serons sur une boite de
 dialogue suivante :

 \begin{figure}[!h]
    \centering
    \includegraphics[scale=0.5]{mesDiagrammes/img/dialohue.png}
    \caption{Interface de paramétrage pour test CEM}
    \label{fig:figure45}
\end{figure}

Nous choisissons la norme CISPR 25 RE TEM, qui est la norme appropriée pour les équipements radars, 
et la classe 1.

Une fois lancé la simulation, nous avons le diagramme suivant décrit par la figure 
\ref{fig:figure46}.
Ce diagramme est un graphique de mesures de compatibilité électromagnétique (CEM), très 
probablement lié à des tests d'émissions ou d'immunité radiofréquence. Il permet 
d'analyser le comportement d'un équipement sur différentes bandes de fréquences.

\begin{figure}[!h]
    \centering
    \includegraphics[scale=0.35]{mesDiagrammes/img/emchomolgation.png}
    \caption{Interface de paramétrage pour test CEM}
    \label{fig:figure46}
\end{figure}

\section{Interprétation du diagramme de test CEM}

% \subsubsection{Axes du graphique}
% \begin{itemize}
%     \item Axe des X (horizontal): Représente la Fréquence (Freq / GHz), allant de 
%     0.0001 GHz (100 kHz) à 1 GHz. Les lignes verticales pointillées indiquent des 
%     points de fréquence spécifiques ou des limites de bande.
%     \item Axe des Y (vertical): Représente la Magnitude (Magnitude / dB(V/m)), 
%     exprimée en décibels par volt par mètre. C'est une mesure de l'intensité du champ électromagnétique. Des valeurs plus faibles sur cet axe indiquent généralement des émissions plus faibles (ce qui est souvent souhaitable pour la CEM).
% \end{itemize}

% Le graphique présente plusieurs séries de données, chacune identifiée par une 
% couleur et un type de marqueur différents, et nommées selon le format 
% "$Type-Bande\_NomBande$".

Après tests nous avons procéder à récolter les mesures suivantes

\subsubsection{Types de mesures}

\begin{itemize}
    \item AVG-Band (Average Band) : Représente les mesures de valeur moyenne.
        % \subitem $AVG-Band\_LW$ et $AVG-Band\_MW$ : Les niveaux se situent autour de 
        % -74 dB(V/m) et -88 dB(V/m) respectivement. Ce sont les plus élevés parmi les 
        % mesures moyennes.

        % \subitem $AVG-Band\_FM$, $AVG-Band\_TVI$, $AVG-Band\_TVIII$ : Les niveaux 
        % sont autour de -90 dB(V/m).

        % \subitem $AVG-Band\_CB$, $AVG-Band\_DAB$, $AVG-Band\_SW$, $AVG-Band\_VHF1$, 
        % $AVG-Band\_VHF2$, $AVG-Band\_VHF3$ : Ces bandes montrent les niveaux moyens les 
        % plus bas, autour de -96 dB(V/m).
    
    \item Peak-Band (Peak Band) : Représente les mesures de valeur de crête (le 
    maximum instantané).
        % \subitem $Peak-Band\_LW$ et $Peak-Band\_MW$ : Atteignent -54 dB(V/m) et 
        % -68 dB(V/m), ce qui représente les magnitudes maximales enregistrées sur ce 
        % graphique.

        % \subitem $Peak-Band\_FM$ : Environ -70 dB(V/m).

        % \subitem $Peak-Band\_CB$, $Peak-Band\_VHF1$, $Peak-Band\_VHF2$, $Peak-Band\_VHF3$, 
        % $Peak-Band\_SW$ : Se situent autour de -76 dB(V/m).

        % \subitem $Peak-Band\_TVI$ et $Peak-Band\_TVIII$ : Environ -80 dB(V/m).

        % \subitem $Peak-Band\_DAB$ : Environ -86 dB(V/m).
    
    \item QP-Band (Quasi-Peak Band) : Représente les mesures de quasi-crête, 
    qui sont souvent utilisées dans les normes CEM pour évaluer l'impact des 
    interférences sur les récepteurs humains (elles sont moins sensibles aux 
    bruits impulsifs que les mesures de crête pures).
        % \subitem $QP-Band\_LW$ : Environ -67 dB(V/m).

        % \subitem $QP-Band\_FM$ : Environ -83 dB(V/m).

        % \subitem $QP-Band\_CB$ : Environ -89 dB(V/m).

\end{itemize}

\subsubsection{Sources potentielles d'interférences}

Les sources potentielles d'interférences sont les bandes de fréquences où les niveaux 
de magnitude (dB(V/m)) sont les plus élevés, en particulier pour les mesures "Peak" 
(crête) et "QP" (quasi-crête), car ce sont souvent ces valeurs qui sont comparées 
aux limites réglementaires.

% En observant le graphique :

% \begin{itemize}
%     \item Bandes LW (Long Wave) et MW (Medium Wave) : Les mesures $Peak-Band\_LW$ et 
%     $Peak-Band\_MW$ affichent les niveaux d'émission les plus élevés, atteignant 
%     environ -54 dB(V/m) et -68 dB(V/m) respectivement. Ces bandes, situées aux 
%     fréquences les plus basses (0.0001 GHz à 0.0018 GHz), présentent les magnitudes 
%     les plus importantes, ce qui pourrait indiquer des émissions significatives dans 
%     ces gammes.
%     \item Bande FM (Frequency Modulation) : Les mesures $Peak-Band\_FM$ montrent 
%     également des niveaux relativement élevés, autour de -70 dB(V/m), dans la gamme 
%     de fréquences de la radio FM (0.076 GHz à 0.108 GHz).
%     \item Bandes CB (Citizen Band), VHF1, VHF2, VHF3, TVI, TVIII : Les mesures 
%     "Peak" dans ces bandes sont généralement autour de -76 dB(V/m) à -86 dB(V/m). 
%     Bien que moins élevées que les bandes LW/MW/FM, elles représentent toujours des 
%     émissions à surveiller.
% \end{itemize}

\subsubsection{Niveaux d'émission de l'équipement}

Les niveaux d'émission sont la magnitude du champ électromagnétique mesurée en dB(V/m) 
pour chaque bande et chaque type de détection. Voici une synthèse des 
niveaux observés dans le graphique :

\subsection{Analyse des niveaux d'émission}

Nous allons commencer par analyser la puissance rayonnée par l'équipement, comme nous le présente la figure \ref{fig:figure47}. 
\begin{figure}[!h]
    \centering
    \includegraphics[scale=0.35]{mesDiagrammes/plotsTest/Power Radiated.png}
    \caption{Puissance rayonnée durant le test CEM}
    \label{fig:figure47}
\end{figure}

Ce graphique de la figure \ref{fig:figure47} illustre la puissance rayonnée (en Watts, partie réelle) en fonction de la fréquence (en GHz).
Il montre que la puissance rayonnée est relativement stable autour de 0.493 W sur une large gamme de fréquences (de 1 GHz à environ 8 GHz).
Un pic d'émission très net est observé juste en dessous de 9 GHz, où la puissance atteint près de 0.498 W. Après ce pic, la puissance diminue brusquement.
Ce graphique  indique les fréquences d'opération principales de l'EUT (le radar lui-même).


\begin{figure}[!h]
    \centering
    \includegraphics[scale=0.35]{mesDiagrammes/plotsTest/VSWR1.png}
    \caption{VSWR durant le test CEM}
    \label{fig:figure48}
\end{figure}

La figure \ref{fig:figure48} quand à elle illustre le rapport d'onde stationnaire de tension (VSWR) durant le test CEM.
Il montre que le VSWR reste généralement en dessous de 2:1 sur la plupart des bandes de fréquence, ce qui est considéré comme acceptable pour 
une bonne adaptation d'impédance. Cependant, des pics occasionnels au-dessus de 2:1 sont observés, en particulier autour de 9 GHz, ce qui 
pourrait indiquer des problèmes d'adaptation d'impédance à ces fréquences.

\begin{figure}[!h]
    \centering
    \includegraphics[scale=0.4]{mesDiagrammes/plotsTest/AVG-Band_LW.png}
    \includegraphics[scale=0.4]{mesDiagrammes/plotsTest/Peak-Band_LW.png}
    \includegraphics[scale=0.4]{mesDiagrammes/plotsTest/QP-Band_LW.png}
    \caption{Niveaux d'émission durant le test CEM}
    \label{fig:figure49}
\end{figure}

Les graphiques de la figure \ref{fig:figure49} montrent les niveaux d'émission pour trois types de mesures : moyenne (AVG), crête (Peak) et quasi-peak (QP).
Le niveau d'émission moyen stable d'environ -74 dB(V/m) entre 0.0001 GHz (100 kHz) et 0.0003 GHz (300 kHz).
et le niveau de crête de l'émission dans cette même bande est plus élevé, atteignant environ -54 dB(V/m). 
La différence significative entre les valeurs moyennes et de crête peut suggérer la présence de signaux impulsifs ou de bruit importants 
dans cette bande.
Le niveau quasi-peak est légèrement plus bas que le niveau de crête, mais reste supérieur au niveau moyen, ce qui est typique pour les mesures CEM.

\section{Rapport d'analyse}

Le rapport d'analyse présente une évaluation détaillée des résultats des tests de rayonnement et de 
compatibilité électromagnétique (CEM) effectués sur l'équipement radar. Les données recueillies ont 
été analysées pour déterminer la conformité de l'équipement aux normes en vigueur.

Les graphiques et tableaux inclus dans ce rapport fournissent une vue d'ensemble des performances de 
l'équipement, mettant en évidence les points forts et les domaines nécessitant des améliorations. 
Les résultats montrent que l'équipement fonctionne dans les limites acceptables pour la plupart 
des bandes de fréquence, bien que des pics d'émission aient été observés à certaines fréquences.

En effet l'annexe 2 présente le modèle de rapport après test, qui peut être 
utilisé comme référence pour les futurs tests de conformité.

L'annexe 3 fournit le modèle de rapport que le technicien de l'ARPTC devra remplir et fournir au demandeur.

L'annexe 4 présente le modèle de rapport que le demandeur devra remplir et fournir 
à l'ARPTC.


\section{Aspect Budgétaire}

\subsection{Coûts Initiaux et d'Investissement (Infrastructure et Équipement)}


\begin{itemize}
    \item Chambre Anéchoïque (pour tests CEM et RF) : La construction et 
    l'installation d'une chambre anéchoïque de petite à moyenne taille 
    (environ 7x7m²) représentent un investissement majeur.

    \subitem Estimation : $108 000$  à $1080000$ USD (ou plus pour des 
    installations très sophistiquées). Ce coût inclut la cage de 
    Faraday, les matériaux absorbants (dièdres pyramidaux), 
    l'installation et l'intégration.

    \subitem Les matériaux absorbants seuls (nos 58 panneaux TSA-200PI).
    \subitem Estimation des matériaux seuls : $32 400$ à $108 000$ USD.

    \item Équipements de Mesure RF (Composants du VNA dédié) :
    \subitem Les modules électroniques de base :
        \begin{itemize}
            \item Module LTC5510 (mélangeur RF) : Environ 54  à
            $216$ USD par unité.
            \item Module ADF4351 (synthétiseur de fréquence) : Environ 32  à
            $162$ USD par unité.
            \item Amplificateur logarithmique AD8317 : Environ 11  à
            $54$ USD par unité.
        \end{itemize}
    \subitem Si l'ARPTC optait pour un Analyseur de Réseau Vectoriel (VNA)
    professionnel tout-en-un du marché, les prix sont bien plus élevés :

    \subitem Estimation VNA professionnel : $10 800$ (entrée de gamme) 
    à $540 000$ USD ou plus (haut de gamme/laboratoire).

    \subitem Estimation VNA professionnel : $10 800$ (entrée de gamme) 
    à $540 000$ USD ou plus (haut de gamme/laboratoire).


\end{itemize}







\subsection{Coûts des Licences Logicielles}
Les logiciels spécialisés sont souvent soumis à des licences coûteuses, surtout pour un usage professionnel.

\begin{itemize}
    \item Qt Creator (licence commerciale) : Estimation : De quelques 
    milliers à plusieurs dizaines de milliers de dollars USD par 
    développeur/poste par an. (Ex: $5 000$ à $30 000$ USD).

    \item CST Studio Suite (pour simulation RF/CEM) : Estimation : 
    $32 400$ à $108 000$ USD ou plus par an pour une suite complète.

    \item COMSOL Multiphysics (pour modélisation de chambre anéchoïque) :
    Estimation : $10 800$ à $54 000$ USD par an (selon les modules).

    \item Antenna Magnus : Estimation de quelques milliers à plusieurs 
    dizaines de milliers de dollars USD. (Ex: $2 000$ à $20 000$ USD).

    \item KiCAD : Généralement gratuit (logiciel open-source).

    \item Total estimé pour l'acquisition initiale de licences 
    logicielles (pour plusieurs postes/modules) : $108 000$ à $540 000$ 
    USD et au-delà.
\end{itemize}

\subsection{Coûts de Formation et de Compétence}
\begin{itemize}
    \item Formation des ingénieurs et techniciens : 
    \subitem Estimation : $2 160$ à $10 800$ USD par personne pour des 
    formations spécialisées et certifiantes (ex: CST, COMSOL, Qt).

    \item Formation continue et mise à jour des compétences : 
    \subitem Estimation : $1 080$ à $5 400$ USD par an par personne.
\end{itemize}

\subsection{Coûts Opérationnels et de Maintenance Annuels}

\begin{itemize}
    \item Maintenance de la chambre anéchoïque :

    \subitem Estimation : $5 400$ à $21 600$ USD par an (nettoyage, 
    recalibrage, remplacement de matériaux absorbants).

    \item Maintenance des équipements de mesure RF :
    \subitem Estimation : $5 400$ à $32 400$ USD par an (calibration, 
    réparations, mises à jour logicielles).

    \item Coûts des licences logicielles (renouvellement) :
    \subitem Estimation : $32 400$ à $108 000$ USD par an.

    \item Coûts de fonctionnement divers (énergie, espace, etc.) :
    \subitem Estimation : $5 400$ à $21 600$ USD par an.
\end{itemize}
\subsection{Récapitulatif Global (Estimations Larges en USD)}
\begin{itemize}
    \item Investissement Initial Total (hors R\&D interne) :
    \subitem Entre $216 000$ et $2 160 000$ USD (voire plus), selon l'ampleur et le niveau de sophistication des installations et des équipements achetés.

    \item Coûts Annuels d'Opération et de Maintien :
    \subitem Entre $43 200$ et $162 000$ USD (voire plus), principalement pour la maintenance, la calibration et les licences logicielles.

\end{itemize}

Ces estimations en dollars américains confirment que l'investissement 
nécessaire est conséquent, mais qu'il est justifié par les retombées 
positives en termes de sécurité, de productivité et de conformité 
réglementaire pour un secteur économique aussi vital que l'exploitation 
minière en RDC.


\section{Conclusion partielle}
Pour conclure, ce chapitre a démontré la capacité opérationnelle de notre système d'homologation en présentant les résultats des 
tests de rayonnement et de compatibilité électromagnétique (CEM). Les diagrammes générés et leur interprétation ont permis de confirmer 
le respect des normes, mettant en évidence les performances de l'équipement radar sous test. Ces résultats valident l'efficacité de l'outil 
développé, soulignant son potentiel à assister l'ARPTC dans le processus d'homologation des systèmes radars miniers et à garantir leur conformité.












     %================================================ CHAPITRE 3 ===============================================
    \addcontentsline{toc}{chapter}{CONCLUSION GENERALE}
\chapter*{CONCLUSION GENERALE}

Le présent travail a abordé une problématique cruciale pour la 
République Démocratique du Congo : l'absence d'un outil d'aide à 
l'homologation des systèmes radars spécifiquement exploités dans 
les sites miniers par l'Autorité de Régulation de la Poste et des 
Télécommunications au Congo (ARPTC). Face aux risques d'accidents 
liés aux mouvements de terrain dans les exploitations minières et 
à la nécessité de garantir la conformité et la sécurité des 
équipements radars utilisés, notre étude a proposé une solution 
concrète et innovante.

Nous avons débuté par une étude approfondie de l'existant, en 
présentant les cadres opérationnels de l'entreprise minière MMG 
Kinsevere et de l'ARPTC, tout en analysant les défis actuels du 
processus d'homologation. Cette phase a permis de spécifier les 
besoins fonctionnels et non fonctionnels de notre futur système, 
jetant ainsi les bases de sa conception.

La conception de l'outil d'homologation s'est appuyée sur des 
méthodologies robustes, telles que le modèle en V et l'approche Model 
Based Design (MBD), pour traduire ces besoins en une architecture 
détaillée. Nous avons défini les diagrammes d'exigences, de cas 
d'utilisation et d'activités, et conçu une interface homme-machine 
intuitive, préparant ainsi le terrain pour la phase d'implémentation.

L'implémentation du système a été minutieusement décrite, couvrant les 
aspects matériels et logiciels. La conception et la simulation d'une 
chambre anéchoïque adaptée, ainsi que le développement d'un analyseur 
de réseaux vectoriel (VNA) dédié, intégrant des composants clés tels 
que les mélangeurs de fréquences, les oscillateurs locaux et les 
amplificateurs logarithmiques, ont constitué le cœur de cette 
réalisation technique. L'application logicielle, développée avec Qt 
Creator, assure l'acquisition, le traitement et la visualisation des 
données, notamment via la mise en œuvre de la transformée de Fourier 
rapide (FFT) et la gestion de la communication UART.

Enfin, les tests et l'interprétation des résultats ont démontré la 
capacité opérationnelle de notre outil. En simulant des mesures RF et 
des tests de compatibilité électromagnétique (CEM) selon la norme CISPR 
25, nous avons validé la pertinence des données obtenues (émissions 
rayonnées en ondes longues, moyennes, courtes, énergie du champ, 
puissance rayonnée) et leur interprétation au regard des critères de 
performance (A, B, C, D). Ces essais confirment que l'outil développé 
est apte à évaluer la conformité et les performances radiofréquences 
des équipements radars, contribuant ainsi à la sécurité des opérations 
minières.

Ce travail ouvre des perspectives significatives pour l'ARPTC, en lui 
fournissant un cadre méthodologique et un outil pratique pour 
l'homologation des systèmes radars, garantissant le respect des 
normes nationales et internationales. Comme perspectives d'avenir, 
nous envisageons 

\begin{itemize}
    \item Développement et Validation d'un Prototype Physique Robuste
        La première et la plus cruciale des perspectives serait de 
        passer de la simulation à la réalisation concrète d'un prototype 
        physique de l'analyseur de réseaux vectoriel (VNA) dédié et de 
        la chambre anéchoïque.

        \begin{itemize}
            \item Tests en conditions réelles : Soumettre le prototype à des tests exhaustifs non seulement en laboratoire, mais aussi dans des environnements contrôlés qui simulent les conditions minières (température, humidité, vibrations, poussière).

            \item Fiabilité et durabilité : Optimiser la conception matérielle pour garantir une robustesse et une fiabilité à long terme, essentielles pour un outil destiné à un usage industriel et réglementaire.

            \item Calibration avancée : Développer des procédures de calibration plus poussées et potentiellement des mécanismes d'auto-calibration pour maintenir la précision des mesures sur le temps.

        \end{itemize}


    \item Amélioration de l'Expérience Utilisateur et de l'Intelligence 
    de l'Interface (IHM) L'interface logicielle développée avec Qt Creator est une base solide. Les perspectives ici se concentrent sur la rendre encore plus accessible et "intelligente" :

    \begin{itemize}
        \item Simplification des rapports : Intégrer des modules de génération de rapports qui traduisent les données techniques complexes en des résumés clairs et des indicateurs visuels de conformité pour les non-experts (décideurs, exploitants miniers).

        \item Aide à la décision intégrée : Ajouter des fonctionnalités d'analyse automatique des résultats, qui pourraient par exemple mettre en évidence les points de non-conformité, suggérer des actions correctives ou générer des alertes spécifiques.

        \item Intelligence Artificielle et Apprentissage Automatique : Explorer l'intégration de l'IA pour l'identification automatique des signatures d'interférences, la prédiction des défaillances d'équipements, ou même l'optimisation des procédures de test basées sur les données historiques.

    \end{itemize}

    \item Gestion Avancée des Données et Intégration Systèmes
    Pour une traçabilité et une utilisation optimale des données, il serait essentiel de :

    \begin{itemize}
        \item Déploiement d'une base de données sécurisée : Mettre en place un système de gestion de base de données robuste pour archiver toutes les données d'homologation (rapports, mesures brutes, historique des équipements) de manière sécurisée et accessible.

        \item Intégration avec les systèmes de l'ARPTC : Développer des API ou des connecteurs pour que l'outil d'homologation puisse s'intégrer fluidement avec d'autres systèmes de l'ARPTC (gestion des licences, suivi des réglementations, etc.).

        \item Tableaux de bord et statistiques : Créer des tableaux de bord interactifs pour l'ARPTC, permettant de visualiser les tendances d'homologation, les types de non-conformité les plus fréquents, ou l'état général du parc d'équipements radars.

    \end{itemize}

    \item Extension de la Portée Réglementaire et Technologique
    Le travail se concentre sur les radars miniers, mais il pourrait être élargi :

    \begin{itemize}
        \item Tests de sécurité électrique approfondis : Comme mentionné en conclusion, l'implémentation complète des tests de sécurité électrique est une perspective directe pour offrir une solution d'homologation exhaustive.

        \item Autres équipements radioélectriques miniers : Adapter l'outil pour l'homologation d'autres systèmes de communication critiques utilisés dans les mines (radios PMR, systèmes IoT, etc.).

        \item Harmonisation Internationale : Poursuivre les efforts pour s'assurer que les méthodes et les normes d'homologation utilisées par l'ARPTC sont non seulement conformes aux normes nationales, mais aussi en phase avec les meilleures pratiques et réglementations internationales (UIT, ETSI, FCC), facilitant ainsi le commerce et l'adoption de technologies.

    \end{itemize}

    \item Modèle Économique et Partenariats Stratégiques
Au-delà des coûts, les perspectives incluent la valorisation économique du projet :

    \begin{itemize}
        \item Partenariats Public-Privé : L'ARPTC pourrait envisager des partenariats avec des entreprises spécialisées dans la fabrication d'équipements de test ou des instituts de recherche pour la production et la maintenance de l'outil.

        \item Services d'expertise : Le centre d'homologation de l'ARPTC pourrait proposer ses services à d'autres pays ou régions confrontés à des défis similaires, devenant un pôle d'expertise en Afrique.
    \end{itemize}

\end{itemize}

Ce projet constitue également une 
référence pour les entreprises minières et les futurs chercheurs, 
soulignant l'importance de la rigueur scientifique et de l'innovation 
technologique pour la sécurité et le développement durable du secteur.

    %=============================================== BIBLIOGRAPHIE ===============================================
    \cleardoublepage % Pour s'assurer que la numérotation reprend à la page suivante
    
    \nocite{*}
    \addcontentsline{toc}{chapter}{BIBLIOGRAPHIE}
    \bibliographystyle{plain}
    \bibliography{myBib}

    \newpage

    \appendix
    \addcontentsline{toc}{chapter}{ANNEXES}
    \chapter*{ANNEXES}

    % \includepdf[pages={12, 13}, scale=0.8]{Décision-065-2019-portant-modification-de-la-decision-024_2006-relative-à-la-directive-fixant-le-régime-dhomologation-des-équipements-et-installations-des-télécommunications.pdf}  % Insertion de toutes les pages

    \includepdf[pages=-]{formulaire_demande_homologation.pdf}  % Insertion de toutes les pages

    \includepdf[pages=-]{formulaire_client.pdf}  % Insertion de toutes les pages
    
    \includepdf[pages=-]{formulaireInform.pdf}  % Insertion de toutes les pages
\end{document}




