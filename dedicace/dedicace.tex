\addcontentsline{toc}{chapter}{DÉDICACE}
    \chapter*{DÉDICACE}
La dédicace d'un travail scientifique est une courte inscription placée au début de l'ouvrage 
(souvent après la table des matières) pour rendre hommage à une ou plusieurs personnes 
(famille, amis, mentors) qui ont apporté un soutien moral ou intellectuel crucial, tout en étant différente 
des remerciements plus formels qui gratifient l'ensemble des collaborateurs académiques et professionnels, 
marquant ainsi la dimension humaine derrière le projet scientifique.

En quoi consiste-t-elle ?
\begin{itemize}
    \item Un hommage personnel : C'est un espace pour exprimer gratitude et affection à ceux qui vous ont soutenu 
                                (parents, conjoint, amis) durant les longs mois de recherche, apportant ainsi une touche 
                                personnelle et chaleureuse à l'œuvre.
    \item Une inscription succincte : Elle doit être brève, souvent écrite en italique et placée en haut à droite de la page, 
                                selon des conventions de mise en page spécifiques.
    \item Un acte symbolique : Elle matérialise le lien entre l'auteur et son œuvre, attestant de sa présence et de 
                                son parcours, et se distingue des remerciements qui listent les contributions techniques et 
                                professionnelles. 
\end{itemize}
