%================================================ CHAPITRE 0 ===============================================
\chapter{INTRODUCTION GÉNÉRALE}
\section{Contexte du travail}
    Etablir le contexte de votre travail de recherche est essentiel pour situer votre étude dans son environnement académique, 
    scientifique et pratique.
    Il s'agit de présenter les circonstances, les motivations et les enjeux qui ont conduit à la réalisation de ce travail.
    Le contexte doit inclure une revue succincte de la littérature existante, les lacunes
    identifiées, ainsi que les besoins spécifiques auxquels votre recherche vise à répondre.
    En définissant clairement le contexte, vous permettez au lecteur de comprendre l'importance et
    la pertinence de votre travail dans le domaine étudié.

\section{Problématique}
\paragraph{}
    La problématique d'un travail scientifique est une question centrale qui guide la recherche. Elle identifie un 
    problème spécifique ou un défi dans le domaine d'étude, en soulignant son importance et ses
    implications. La problématique doit être formulée de manière claire et concise, en mettant en évidence les
    lacunes dans les connaissances actuelles ou les besoins non satisfaits. Elle sert de fondement à la définition des objectifs 
    de la recherche et oriente la méthodologie employée pour y répondre.

\section{Hypothèse}
\paragraph{}
    Les hypothèses dans un travail scientifique sont des propositions ou des suppositions formulées avant la 
    réalisation de la recherche. Elles servent de points de départ pour l'investigation et guident
    la collecte et l'analyse des données. Les hypothèses doivent être claires, testables et basées sur une
    compréhension préalable du sujet. Elles permettent de structurer la recherche en définissant ce que l'on
    s'attend à découvrir ou à démontrer, et elles sont essentielles pour évaluer les résultats obtenus.
\section{Choix et intérêt du sujet}
\paragraph{}
    Justifier le choix et l'intérêt du sujet de votre travail scientifique est crucial pour démontrer sa pertinence et son impact potentiel.
    Il s'agit d'expliquer pourquoi ce sujet a été sélectionné, en mettant en avant son importance dans le domaine d'étude, les
    lacunes qu'il vise à combler, et les bénéfices attendus de la recherche
\section{Méthodologies et techniques}
    \subsection{Méthodologies}
    \paragraph{}
    Enumérer et décrire les méthodologies que vous avez adoptées pour mener à bien votre travail scientifique est essentiel pour assurer la rigueur et la crédibilité de votre recherche.
    Voici quelques méthodologies couramment utilisées dans les travaux scientifiques :
    \begin{itemize}
        \item \textbf{Méthode expérimentale} : Implique la manipulation de variables pour observer leurs effets, souvent utilisée dans les sciences naturelles et l'ingénierie.
        \item \textbf{Méthode descriptive} : Consiste à collecter des données pour décrire un phénomène ou une situation sans intervenir, couramment utilisée en sciences sociales.
        \item \textbf{Méthode comparative} : Compare différents groupes ou conditions pour identifier des différences ou des similitudes, utile dans diverses disciplines.
        \item \textbf{Méthode analytique} : Décompose un problème complexe en ses éléments constitutifs pour mieux le comprendre, souvent utilisée en mathématiques et en informatique.
        \item \textbf{Méthode qualitative} : Se concentre sur la compréhension des phénomènes à travers des données non numériques, telles que les entretiens et les observations.
        \item \textbf{Méthode quantitative} : Utilise des données numériques et des analyses statistiques pour tester des hypothèses et mesurer des variables.
        \item \textbf{Méthode mixte} : Combine des approches qualitatives et quantitatives pour bénéficier des avantages des deux méthodologies.
    \end{itemize}

    \subsection{Techniques}
    Enumerer et décrire les techniques spécifiques que vous avez utilisées pour collecter, analyser et interpréter les données dans votre travail scientifique est crucial pour assurer la transparence et la reproductibilité de votre recherche.
    Voici quelques techniques couramment employées dans les travaux scientifiques :
    \begin{itemize}
        \item \textbf{Enquêtes et questionnaires} : Utilisés pour collecter des données auprès d'un large échantillon de participants.
        \item \textbf{Entretiens} : Permettent d'obtenir des informations détaillées et qualitatives en interrogeant des individus ou des groupes.
        \item \textbf{Observation participante} : Implique l'immersion dans un environnement pour observer les comportements et les interactions.
        \item \textbf{Analyse statistique} : Utilisée pour traiter et interpréter des données quantitatives à l'aide de logiciels statistiques.
        \item \textbf{Modélisation informatique} : Permet de simuler des phénomènes complexes à l'aide de modèles mathématiques et informatiques.
        \item \textbf{Expérimentation en laboratoire} : Implique la réalisation d'expériences contrôlées pour tester des hypothèses spécifiques.
        \item \textbf{Analyse de contenu} : Technique qualitative utilisée pour analyser des documents, des textes ou des médias afin d'identifier des thèmes ou des patterns.
        \item \textbf{Techniques de visualisation des données} : Utilisées pour représenter graphiquement les données afin de faciliter leur interprétation.
    \end{itemize}

\section{État de l'art}
\paragraph{}
    Nous ne pouvons pas prétendre être le premier à faire des investigations dans ce domaine. Cepandant, UDBL nous recommande 
    de repertorié les travaux de fin d'étude qui ont été réalisés dans le même domaine que le vôtre. Ces travaux font reference
    aux étudiants qui vous ont précédé à UDBL/ESIS

    
\section{Délimitation du travail}
        Pour qu'un travail scientifique aboutisse et qu'il réponde à la demande de la société, il doit être projeté dans le temps et dans l'espace
    \begin{itemize}
        \item Limites temporelles : C'est le temps imparti pour la réalisation de ce travail, qui est généralement de six (6) mois, là
        que vous aurez planifié les différentes étapes de votre recherche, y compris la collecte de données, l'analyse, la conception, 
        l'implémentation et la rédaction du rapport final.
        \item Limite spatiale : C'est le lieu où se déroule notre étude. Par exemple, si votre travail porte sur l'homologation des systèmes radars dans les sites miniers,
        vous pouvez spécifier que votre étude se concentre sur une région géographique particulière ou sur un type spécifique de site minier.
    \end{itemize}

\section{Subdivision du travail}
    Subdiviser le travail scientifique en chapitres clairs et logiques est essentiel pour structurer la présentation de la recherche.
    
\section{Logiciels et équipements utilisés}
    Pour pouvoir élaborer notre travail, nous avons utilisé les outils suivants :
    \begin{itemize}
        \item \textbf{Latex} : Pour produire les documents annexes ainsi que les documents d'homo\-logation
        \item \textbf{Microsoft Office Excel} : pour produire les graphiques
        \item \textbf{Draw IO} : le logiciel de représentation nous permettant de faire des architectures et des organigrammes.
        \item \textbf{Visual Studio Code} : Un éditeur de texte.
    \end{itemize}